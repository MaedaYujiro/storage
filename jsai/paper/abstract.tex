%%%%%%%%%%%%%%%%%%%%%%%%%%%%%%%%%%%%%%%%%%%%%%%%%%%%%%%%%% 
\section*{概要}
%%\pagenumbering{roman}
%%%%%%%%%%%%%%%%%%%%%%%%%%%%%%%%%%%%%%%%%%%%%%%%%%%%%%%%%% 

極小・極大モデルとは,与えられた命題論理式のモデル集合に包含関係を導入
して定義されるものであり,真が割り当てられる変数集合をこれ以上減らせな
い,あるいは増やせないモデルである.
%
従来の計算方法はモデルを複数求めながら徐々に小さく・大きくしていく方法であり,
SATソルバーの呼び出し回数が多くなることが問題であった.
%
本研究では,CNFの各リテラルに着目し,
他リテラルで節が充足される場合に当該変数の真偽を強制する制約を追加することで,
求めたモデルが元のCNFの極小・極大モデルと一致するCNF変換を提案・実装した.
$3 \times n$のグリッドグラフに基づく極小支配集合・極大独立集合問題で評価した結果,
提案手法は呼び出し回数を大幅に削減し,
従来方法より多くの問題を制限時間内に解けることを確認した.

% 極小・極大モデルとは,与えられた命題論理式のモデル集合に包含関係を導入
% して定義されるものであり,真が割り当てられる変数集合をこれ以上減らせな
% い,あるいは増やせないモデルである.この意味において,極小・極大モデル
% は冗長性のない代表的なモデルであり,グラフ理論や多目的最適化におけるパ
% レートフロントの計算など,様々な応用が知られている.

% 既存の計算方法では,SATソルバーを繰り返し起動しながら,得られたモデル
% を徐々に小さく,もしくは大きくしていき,最終的に充足不能性を判定するこ
% とで極小性・極大性を保証していた.しかし,この方法ではSATソルバーの複
% 数回起動や,計算コストの高い充足不能性判定がボトルネックになるという問
% 題があった.

% 本発表では,SAT符号化を用いて命題論理式を適切に変換することで,元の論
% 理式の極小・極大モデルを普通のモデルとして直接得る計算方法を提案する.
% これにより,SATソルバーの起動回数を抑え,充足不能性判定を回避できる.
% 極小支配集合および極大独立集合をベンチマークとした比較実験の結果,既存
% 手法よりも高速に極小・極大モデルを計算できることを確認した.