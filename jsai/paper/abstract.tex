%%%%%%%%%%%%%%%%%%%%%%%%%%%%%%%%%%%%%%%%%%%%%%%%%%%%%%%%%% 
\section*{概要}
%%\pagenumbering{roman}
%%%%%%%%%%%%%%%%%%%%%%%%%%%%%%%%%%%%%%%%%%%%%%%%%%%%%%%%%% 

命題論理式の極小モデル・極大モデルとは,モデル集合に包含関係を導入し,真の変数集合をこれ以上減らせない・増やせないモデルである.従来の反復起動法はモデル候補ごとにSATソルバーの起動が増え計算コストが大きい.本研究では,CNFの各リテラルに着目し,他リテラルで節が充足される場合に当該変数の真偽を定める制約を追加することで,SATソルバー1回の起動で1つの極小・極大モデルを得るCNF変換を提案・実装した.$3 \times n$のグリッドグラフに基づく極小支配集合・極大独立集合問題で評価した結果,提案手法はSATソルバーの起動回数を大幅に削減し,反復起動法より多くの問題を制限時間内に解けることを確認した.

