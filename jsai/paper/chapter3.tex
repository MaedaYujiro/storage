%%%%%%%%%%%%%%%%%%%%%%%%%%%%%%%%%%%%%%%%%%%%%%%%%%%%%%%%%% 
\section{SATソルバー複数回起動による極小・極大モデル計算の既存研究} \label{chap:koshimura}
%%%%%%%%%%%%%%%%%%%%%%%%%%%%%%%%%%%%%%%%%%%%%%%%%%%%%%%%%% 

本章では,論文\cite{koshimura2009minimal}で提案された,SATソルバーを複数回起動することでSAT問題の極小モデルを求める方法について説明する.
ここでSAT問題を符号化した$P$が与えられたとき,極小モデルは以下の手順で求めることができる.
\begin{enumerate}
 \item $P$にSATソルバーを起動し,モデルMを求める
 \item 1.で求めたモデルMのうち,真として含まれるリテラルを$x_1,...,x_m$とし,偽として含まれるリテラルを$y_1,...,y_n$とし,次のF1,F2を作る
  \begin{center}
  F1$~=~\neg (x_1 \land ... \land x_m)$\\
  F2$~=~\neg y_1 \land ... \land \neg y_n$
  \end{center}
 \item $P \coloneq  P \land F1 \land F2$として,SATソルバーを起動する
 \begin{itemize}
  \item UNSATならば,モデルMが極小モデル
  \item SATならば,1.に戻る
 \end{itemize}
\end{enumerate}
アルゴリズムで記述するとAlgorithm\ref{alg:minimal-model}のようになる.

\begin{algorithm}[t]
 \caption{SATソルバー複数回起動による極小モデル生成}
 \label{alg:minimal-model}
 \begin{algorithmic}[1]
  \Require SAT問題の符号化 $P$
  \Ensure $P$ の極小モデル $M$(存在しなければ UNSAT)

  \While{\textsc{solve}$(P)$ が SAT を返し,モデル $M$ を得る}
    \State $X \gets \{x \mid x \text{ は } M \text{ において真となるリテラル}\}$
    \State $Y \gets \{y \mid y \text{ は } M \text{ において偽となるリテラル}\}$
    \State \Comment{$X=\{x_1,\dots,x_m\},\;Y=\{y_1,\dots,y_n\}$ とする}
    \State $F1 \gets \neg(x_1 \land \cdots \land x_m)$ \Comment{$\equiv (\neg x_1 \lor \cdots \lor \neg x_m)$}
    \State $F2 \gets (\neg y_1) \land \cdots \land (\neg y_n)$
    \State $P' \gets P \land F1 \land F2$
    \If{\textsc{solve}$(P')$ が UNSAT}
        \State \textbf{return} $M$
    \Else
        \State $P \gets P'$
    \EndIf
  \EndWhile
  \State \textbf{return} UNSAT
 \end{algorithmic}
\end{algorithm}

\begin{example}\rm
 \ref{sec:defCNF}節の例\ref{ex:CNF}で与えられた式を$P$として,先ほどのアルゴリズムに沿って極小モデルを求めてみる.\\
(1回目のループ)
\begin{enumerate}
 \item $P$にSATソルバーを起動し,$(x_1,x_2,x_3)=(1,1,1)$がモデルMとして求められる
 \item 1.で求めたモデルMから次のF1,F2を作る
  \begin{center}
  F1$~=~\neg (x_1 \land x_2 \land x_3)$\\
  F2$~=~\top$
  \end{center}
 \item $P \coloneq P ~\land ~\neg (x_1 \land x_2 \land x_3)~ \land ~\top$として,SATソルバーを起動すると,SATであるので,1.に戻る.
\end{enumerate}
(2回目のループ)
\begin{enumerate}
 \item $P \coloneq P ~\land ~\neg (x_1 \land x_2 \land x_3)~ \land ~\top$にSATソルバーを起動すると,$(x_1,x_2,x_3)=(1,1,0)$がモデルMとして求められる.
 \item 1.で求めたモデルMから次のF1,F2を作る
  \begin{center}
  F1$~=~\neg (x_1 \land x_2)$\\
  F2$~=~\neg x_3$
  \end{center}
 \item $P \coloneq P \land \neg (x_1 \land x_2) \land \neg x_3$として,SATソルバーを起動すると,UNSATであるので$(x_1,x_2,x_3)=(1,1,0)$は極小モデルである.
\end{enumerate}
以上のようにして,極小モデルが求まる.\\
ただし,上の例の1.で求まるモデルMはいくつかあり,あくまで一例である.上のような操作を繰り返し行うことで,全ての極小モデルを列挙することができる.\\
また,先ほどの手順2.におけるF1,F2を以下のように変更すれば,極大モデルを求めることができる.
\begin{center}
  F1$~=~\neg (\neg y_1 \land ... \land \neg y_n)$\\
  F2$~=~x_1 \land ... \land x_m$
\end{center}
極小モデルでは,いま見つけたモデルMより真となるリテラルが少ないモデルが存在するかをSATソルバーで調べ、存在しなければモデルMを極小モデルとして返すという計算方法であったので,極大モデルでは,いま見つけたモデルMより偽となるリテラルが少ないモデルが存在するかをSATソルバーで調べ、存在しなければモデルMを極大モデルとして返すという計算方法にしている.\\

% $(x_1,x_2,x_3)=(1,1,1)$がモデルMとして求められたとき,$F1=\neg (x_1 \land x_2 \land x_3)$,$F2=\top$となり,$P \coloneq P ~\land ~\neg (x_1 \land x_2 \land x_3)~ \land ~\top$となる.これはSATであるので,1.に戻り,新たに定義した$P$についてSATソルバーを起動すると,$(x_1,x_2,x_3)=(1,1,0)$または,$(x_1,x_2,x_3)=(1,0,1)$を返す.$(x_1,x_2,x_3)=(1,1,0)$が返ってきたとし,このモデルをMとすると,$P$は新たに$P \coloneq P \land \neg (x_1 \land x_2) \land \neg x_3$となる.これはUNSATであるので極小モデルは$(x_1,x_2,x_3)=(1,1,0)$と求まる.

\end{example}

