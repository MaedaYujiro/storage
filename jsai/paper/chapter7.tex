%%%%%%%%%%%%%%%%%%%%%%%%%%%%%%%%%%%%%%%%%%%%%%%%%%%%%%%%%% 
\section{おわりに} \label{chap:conclusion}
%%%%%%%%%%%%%%%%%%%%%%%%%%%%%%%%%%%%%%%%%%%%%%%%%%%%%%%%%%

本論文では,SAT 符号化に基づく CNF 変換を用い,SAT ソルバーを 1 回起動するだけで命題論理式の極小モデル・極大モデルを計算する枠組みを整理した.
極小モデルについては既存の変換手法を実装し,極大モデルについては負リテラルに着目した同様の発想による CNF 変換方法を新たに提案・実装した.

3 行 $n$ 列の Grid graph に基づくベンチマークを用いて,提案手法(trans)と従来の反復起動法(basic)の性能を比較した.
極小支配集合問題(MDS)では,basic が $3\times10$ までしか解けなかったのに対し,trans は $3\times12$ まで解くことができた.
極大独立集合問題(MIS)では,basic が $3\times13$ までであったのに対し,trans は $3\times18$ まで解くことができた.
いずれの問題設定においても,SAT ソルバーの起動回数が大幅に削減され,計算時間の短縮と解ける問題サイズの拡大が確認できた.
以上の結果から,SAT 符号化による CNF 変換が極小・極大モデル計算の実用性を高める有効なアプローチであることを示した.

今後の課題として,まず,グラフ問題以外の制約充足問題や計画問題など,異なる問題領域に対しても同様の傾向が成り立つかを検証する必要がある.
また,変換に伴う補助変数・節の増加による CNF サイズの増大が性能に与える影響を分析し,冗長制約の削減や Tseitin 変換の設計最適化などにより変換そのものを効率化する余地がある.
さらに,CaDiCaL 以外のソルバーでの再現性確認や,特定の変数集合のみを極小化・極大化する部分集合最適化への拡張など,極小・極大モデル計算をより実用的なものに発展させていくことが期待される.
