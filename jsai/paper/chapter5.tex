%%%%%%%%%%%%%%%%%%%%%%%%%%%%%%%%%%%%%%%%%%%%%%%%%%%%%%%%%% 
\section{提案手法の実装} \label{chap:maxminalgo}
%%%%%%%%%%%%%%%%%%%%%%%%%%%%%%%%%%%%%%%%%%%%%%%%%%%%%%%%%% 

\section{変換アルゴリズム}
\label{sec:algo}
極小変換,極大変換の一般的なアルゴリズムをAlgorithm\ref{alg:trans-minmax}に記述する.アルゴリズムでは,入力をCNF$\Phi$とmin,maxとし,入力によって極小変換と極大変換に分岐できる.また出力は変換後のCNF$\Omega$である.

CNF $\Phi$ から変換後の CNF $\Omega$ を構成する手順を,行番号に沿って説明する.Algorithm \ref{alg:trans-minmax} は入力として CNF $\Phi$ と mode $\in\{\mathrm{min},\mathrm{max}\}$ を受け取り,補助制約 $M$ を生成したうえで $\Omega \gets \Phi \land M$ を返す(行1--2, 28--29).

\paragraph{準備(行1--2)}
まず $\Phi$ に出現する命題変数全体を $P \gets \mathrm{Var}(\Phi)$ として取り出す(行1).続いて,追加制約を保管するための CNF を $M \gets \top$ と初期化する(行2).

\paragraph{mode による分岐(行3--11)}
各変数 $p \in P$ について,その変数に関する「極小/極大らしさ」を強制する制約を追加する(行3).このとき,まず $c_1 \gets \bot$ を用意し(行4),mode に応じて 2 つのリテラル $t$ と $h$ を次のように選ぶ(行5--11).
\begin{itemize}
  \item mode = min のとき: $t \gets p,\ \ h \gets \neg p$(行6--7).
  \item mode = max のとき: $t \gets \neg p,\ \ h \gets p$(行9--10).
\end{itemize}
ここで $t$ は「節の中で注目するリテラル」,$h$ は最終的に追加する含意 $(h \lor c_1)$ の左側に出るリテラル(行26)である.この入れ替えにより,極小変換と極大変換を同一の枠組みで処理できる.

\paragraph{各節に対する補助変数の導入(行12--25)}
次に $\Phi$ の各節 $clause \in \Phi$ を走査し(行12),その節が $t$ を含む場合のみ処理を行う(行13).\\
$t \in clause$ であるとき,新しい補助変数 $x$ を 1 つ導入し(行14),さらに $c_2 \gets \bot$ を初期化する(行15).\\
その後,節 $clause$ 内の各リテラル $l \in clause$ を走査し(行16),$l \neq t$ のものだけを用いて以下を行う(行17--20).\\
\begin{enumerate}
  \item 制約 $(\neg x \lor \neg l)$ を $M$ に追加する(行18).
  \item $c_2 \gets c_2 \lor l$ として,$t$ 以外のリテラルの選言 $c_2$ を作る(行19).
\end{enumerate}
このとき,$(\neg x \lor \neg l)$ は $x \to \neg l$ を表すため,$x$ が真なら「$t$ 以外のリテラルはすべて偽である」ことを要求する.

ループ後(行21),さらに $(c_2 \lor x)$ を $M$ に追加する(行22).
これは $(\neg c_2) \to x$ を表し,節中の他リテラル($t$ 以外)がすべて偽(すなわち $c_2$ が偽)なら $x$ を真にせよ,という意味になる.結果として,$x$ は
\[
x \ \leftrightarrow\  \bigwedge_{l \in clause,\ l\neq t} \neg l
\]
(「その節で $t$ 以外がすべて偽である」こと)を CNF で表す指示変数として機能する.最後に $c_1 \gets c_1 \lor x$ として(行23),$t$ を含む各節について生成した $x$ の選言を $c_1$ に保管する(行23).

\paragraph{変数ごとの主要制約の追加(行26)}
全節の走査が終わると,$M \gets M \land (h \lor c_1)$ を追加する(行26).\\
$c_1$ は「$t$ を含むどれかの節で,$t$ 以外がすべて偽になる(=その節が $t$ に依存する)」ことを表す.したがって $(h \lor c_1)$ は
\[
\neg h \to c_1
\]
を意味し,\textbf{$h$ を偽にする(min なら $p$ を真にする,max なら $p$ を偽にする)場合には,その選択が必須である状況($c_1$ が真)を要求する}
という形で,極小性,極大性を CNF 制約として埋め込む.

\paragraph{出力(行28--29)}
以上で得た $M$ を元の CNF に連言し,$\Omega \gets \Phi \land M$ を返す(行28--29).\\
この $\Omega$ を SAT ソルバーで 1 回解くことで,mode に応じた極小モデル,極大モデルに対応する解を得る(Algorithm 2 の目的).


\begin{algorithm}[t]
\caption{CNF $\Phi$ からCNF への変換方法$\Omega$(min/max)}
\label{alg:trans-minmax}
\begin{algorithmic}[1]
\Require CNF $\Phi$, mode $\in \{\mathrm{min},\mathrm{max}\}$
\Ensure  CNF $\Omega$
\State $P \gets \mathrm{Var}(\Phi)$
\State $M \gets \top$ %\Comment{additional constraints (empty conjunction)}
\ForAll{$p \in P$}
  \State $c_1 \gets \bot$ %\Comment{a disjunction (empty disjunction)}
  \If{$mode = \mathrm{min}$}
    \State $t \gets p$ %\Comment{focus on positive literal $p$ (Eq.1)}
    \State $h \gets \neg p$ %\Comment{final clause is $(h \vee c_1) = (\neg p \vee c_1)$}
  \Else
    \State $t \gets \neg p$ %\Comment{focus on negative literal $\neg p$ (Eq.2)}
    \State $h \gets p$ %\Comment{final clause is $(h \vee c_1) = (p \vee c_1)$}
  \EndIf

  \ForAll{$clause \in \Phi$}
    \If{$t \in clause$}
      \State $x \gets$ new variable
      \State $c_2 \gets \bot$ %\Comment{$c_2 = \bigvee (clause \setminus \{t\})$}
      \ForAll{$l \in clause$}
        \If{$l \neq t$}
          \State $M \gets M \wedge (\neg x \vee \neg l)$ %\Comment{$x \rightarrow \neg l$}
          \State $c_2 \gets c_2 \vee l$
        \EndIf
      \EndFor
      \State $M \gets M \wedge (c_2 \vee x)$ %\Comment{$\neg c_2 \rightarrow x$}
      \State $c_1 \gets c_1 \vee x$
    \EndIf
  \EndFor

  \State $M \gets M \wedge (h \vee c_1)$
\EndFor
\State $\Omega \gets \Phi \wedge M$
\State \textbf{return} $\Omega$
\end{algorithmic}
\end{algorithm}

%=========================
\section{変換アルゴリズムの動作例}
\label{subsec:example-trans-minmax}

本節では,Algorithm~\ref{alg:trans-minmax} が具体的にどのように新規変数を追加し,CNFを変換するかを,小さなサイズの CNF を入力として示す.

\paragraph{入力 CNF}
次の CNF を入力 $\Phi$ とする:
\begin{align}
\Phi \;=\;& (p \lor q)\ \land\ (p \lor \neg q)\ \land\ (\neg p \lor r).
\label{eq:phi-example}
\end{align}
出現変数は $P=\mathrm{Var}(\Phi)=\{p,q,r\}$ である.

%------------------------------------------------------------
\paragraph{極小変換(mode=min)}
Algorithm~\ref{alg:trans-minmax} は各 $p\in P$ について,$t$ を含む節ごとに新規変数を導入し,最後に $(h\lor c_1)$ を追加する.
以下では,変数ごとに追加される新規変数と制約を明示する.\\
また,行26で各変数$p$に対して追加する節$(h \lor c_1)$を以後,極小(極大)性制約と呼ぶ.

%=========================
\subsubsection*{(1) 変数 $p$ に対する処理(min:$t=p,\ h=\neg p$)}

\noindent
$\Phi$ のうち $t=p$ を含む節は
\begin{center}
 $~~~~~~C_1=(p\lor q)$\\
 $\qquad C_2=(p\lor \neg q)$

\end{center}

の 2 つである.それぞれに対して新規変数を導入する

\begin{itemize}
  \item 節 $C_1=(p\lor q)$ に対して新規変数 $x_{p,1}$ を導入する.
  この節で $t=p$ 以外のリテラルは $q$ だけなので,
  $x_{p,1}$ は$q$ が偽であることを表す指示変数($x_{p,1}\leftrightarrow \neg q$)になる.
  Algorithm~\ref{alg:trans-minmax}(行18,22)に従い,次を $M$ に追加する
  \begin{align}
    (\neg x_{p,1}\lor \neg q)\ \land\ (q \lor x_{p,1}). \label{eq:add-xp1}
  \end{align}

  \item 節 $C_2=(p\lor \neg q)$ に対して新規変数 $x_{p,2}$ を導入する.
  $t=p$ 以外のリテラルは $\neg q$ だけなので,
  $x_{p,2}\leftrightarrow \neg(\neg q)\equiv q$ を表す($x_{p,2}\leftrightarrow q$).
  よって追加制約は
  \begin{align}
    (\neg x_{p,2}\lor \neg(\neg q))\ \land\ (\neg q \lor x_{p,2})
    \;\equiv\;
    (\neg x_{p,2}\lor q)\ \land\ (\neg q \lor x_{p,2}). \label{eq:add-xp2}
  \end{align}
\end{itemize}

\noindent
このとき,$c_1$ は,$p$ を含むどれかの節で,$p$ 以外がすべて偽を表す選言なので
\[
c_1 = x_{p,1}\ \lor\ x_{p,2}.
\]
最後に Algorithm~\ref{alg:trans-minmax}(行26)により極小性制約 $(h\lor c_1)$ を追加する
\begin{align}
(\neg p \lor x_{p,1}\lor x_{p,2}). \label{eq:add-p-main}
\end{align}

%=========================
\subsubsection*{(2) 変数 $q$ に対する処理(min:$t=q,\ h=\neg q$)}

\noindent
今度は $t=q$ を含む節だけを見る.$\Phi$ で $q$ を含むのは
\[
C_1=(p\lor q)
\]
のみである.したがって新規変数は 1 つ

\begin{itemize}
  \item 節 $C_1=(p\lor q)$ に対して新規変数 $x_{q,1}$ を導入する.
  $t=q$ 以外のリテラルは $p$ だけなので,$x_{q,1}\leftrightarrow \neg p$ を表す.
  追加制約は
  \begin{align}
    (\neg x_{q,1}\lor \neg p)\ \land\ (p \lor x_{q,1}). \label{eq:add-xq1}
  \end{align}
\end{itemize}

\noindent
このとき $c_1=x_{q,1}$ であり,極小性制約は
\begin{align}
(\neg q \lor x_{q,1}). \label{eq:add-q-main}
\end{align}

%=========================
\subsubsection*{(3) 変数 $r$ に対する処理(min:$t=r,\ h=\neg r$)}

\noindent
$\Phi$ で $r$ を含むのは
\[
C_3=(\neg p\lor r)
\]
のみである.新規変数を 1 つ導入する:

\begin{itemize}
  \item 節 $C_3=(\neg p\lor r)$ に対して新規変数 $x_{r,1}$ を導入する.
  $t=r$ 以外のリテラルは $\neg p$ なので,$x_{r,1}\leftrightarrow \neg(\neg p)\equiv p$ を表す.
  追加制約は:
  \begin{align}
    (\neg x_{r,1}\lor \neg(\neg p))\ \land\ (\neg p \lor x_{r,1})
    \;\equiv\;
    (\neg x_{r,1}\lor p)\ \land\ (\neg p \lor x_{r,1}). \label{eq:add-xr1}
  \end{align}
\end{itemize}

\noindent
このとき $c_1=x_{r,1}$ であり,極小性制約は:
\begin{align}
(\neg r \lor x_{r,1}). \label{eq:add-r-main}
\end{align}

%------------------------------------------------------------
\paragraph{得られる追加制約 $M$ と出力 $\Omega$}
以上で生成された追加制約 $M$ は
\begin{align}
M \;=\;&
(\neg x_{p,1}\lor \neg q)\land(q\lor x_{p,1})
\land(\neg x_{p,2}\lor q)\land(\neg q\lor x_{p,2})
\land(\neg p\lor x_{p,1}\lor x_{p,2})
\nonumber\\
&\land(\neg x_{q,1}\lor \neg p)\land(p\lor x_{q,1})
\land(\neg q\lor x_{q,1})
\nonumber\\
&\land(\neg x_{r,1}\lor p)\land(\neg p\lor x_{r,1})
\land(\neg r\lor x_{r,1}).
\label{eq:M-example}
\end{align}
したがって出力は $\Omega=\Phi\land M$ である:
\begin{align}
\Omega \;=\;& (p \lor q)\land(p \lor \neg q)\land(\neg p \lor r)\ \land\ M.
\label{eq:Omega-example}
\end{align}

導入された各新規変数は,対応する節において,注目リテラル $t$ 以外がすべて偽を表す指示変数である(Algorithm~\ref{alg:trans-minmax} 行18--22).
例えば
\[
x_{p,1}\leftrightarrow \neg q,\qquad
x_{p,2}\leftrightarrow q,\qquad
x_{q,1}\leftrightarrow \neg p,\qquad
x_{r,1}\leftrightarrow p
\]
となっており,極小性制約
\[
(\neg p \lor x_{p,1}\lor x_{p,2}),\quad
(\neg q \lor x_{q,1}),\quad
(\neg r \lor x_{r,1})
\]
は$p,q,r$ を真にする場合は,それが必須であることを $x$ によって証明しなければならないという意味を持ち,CNF に埋め込むことで極小性を与える.

%------------------------------------------------------------
\paragraph{極大変換(mode=max )}
極大変換では,各変数ごとに
\[
t\gets \neg p,\ \ h\gets p
\]
のように $t$ と $h$ が入れ替わるだけで,節の走査・新規変数の導入・制約生成は同一である
(Algorithm~\ref{alg:trans-minmax} 行5--11).
したがって,上の例をそのまま用い,
$\neg p$ を含む節だけが $p$ の処理対象になる点に注意して同様に展開できる.
%=========================


% \section{変換の実装}
% \label{sec:encoding}
% \cite{koshimura2009minimal}で提案された極大・極小モデルの計算方法,\cite{adachi2023}で提案された極小モデルの計算方法,さらに本論文での極大モデルの計算方法を実装するにあたり,どのように実装したかを図\ref{fig:archi}に示す.\\
% 今回の実装では,グラフ問題を用いて実装を行い,極大モデルは極大独立集合問題(MIS),極小モデルは極小支配集合問題(MDS)で計算した.また,CSPの構築にはcp4rust用い,SAT変換は順序符号化,SATソルバーはCaDiCaLを使用した.

% \begin{figure}[ht]
%   \centering
%   \resizebox{\textwidth}{!}{%
%     \begin{tikzpicture}[
  font=\normalsize,
  >=Stealth,
  node distance=14mm and 22mm,
  module/.style={
    rectangle, rounded corners,
    draw=blue!60, thick,
    fill=blue!8,
    align=center,
    minimum width=40mm,
    minimum height=14mm,
    inner sep=6pt
  },
  data/.style={
    rectangle, rounded corners,
    draw=green!60!black, thick,
    fill=green!10,
    align=center,
    minimum width=36mm,
    minimum height=14mm,
    inner sep=6pt
  },
  external/.style={
    rectangle, rounded corners,
    draw=orange!70!black, thick,
    fill=orange!12,
    align=center,
    minimum width=42mm,
    minimum height=14mm,
    inner sep=6pt
  },
  flow/.style={->, thick},
  altflow/.style={->, thick, dashed},
  loop/.style={->, thick, dotted}
]

% ============================================================
% 入力層
% ============================================================
\node[data] (dimacs) {DIMACS\\グラフ};
\node[module, right=of dimacs] (graph) {Graph\\隣接リスト};

\draw[flow] (dimacs) -- (graph);

% ============================================================
% 問題定式化層(左)
% ============================================================
\node[module, right=of graph, yshift=26mm] (mds) {MDS 定式化};
\node[module, right=of graph, yshift=-26mm] (mis) {MIS 定式化};

\draw[flow] (graph) -- (mds);
\draw[flow] (graph) -- (mis);

% ============================================================
% MDS 系(右上ブロック)
% ============================================================
\node[module, right=26mm of mds, yshift=-12mm] (mds_basic)
{MDS\\直接 CNF};

\node[module, right=26mm of mds, yshift=12mm] (mds_trans)
{MDS\\CSP 変換};

\draw[altflow]
  (mds) -- node[midway, above]
  {\cite{koshimura2009minimal}の変換}
  (mds_basic);
\draw[flow]    
  (mds) -- node[midway, above]
  {提案変換}
  (mds_trans);

% ============================================================
% MIS 系(右下ブロック)
% ============================================================
\node[module, right=26mm of mis, yshift=-12mm] (mis_basic)
{MIS\\直接 CNF};

\node[module, right=26mm of mis, yshift=12mm] (mis_trans)
{MIS\\CSP 変換};

\draw[altflow] 
  (mis) -- node[midway, above]
  {\cite{koshimura2009minimal}の変換}
  (mis_basic);
\draw[flow]    
  (mis) -- node[midway, above]
  {\cite{adachi2023}の変換}
  (mis_trans);

% ============================================================
% CSP 層
% ============================================================
\node[module, right=32mm of mds_trans, yshift=-26mm] (csp)
{CSP 構築\\(cp4rust)};

\draw[flow] (mds_trans) -- (csp);
\draw[flow] (mis_trans) -- (csp);

% ============================================================
% エンコード層
% ============================================================
\node[module, right=of csp] (encoder)
{SAT 変換\\OrderEncoderLe};

\draw[flow] (csp) -- (encoder);

% ============================================================
% SAT ソルバ層
% ============================================================
\node[external, below=24mm of encoder] (solver)
{CaDiCaL\\all\_solutions()};

\node[module, right=of solver] (blocking)
{ブロッキング節\\$\neg$ 現在解};

\draw[flow] (encoder) -- (solver);
\draw[flow] (mds_basic) -- (solver);
\draw[flow] (mis_basic) -- (solver);
\draw[loop] (solver) -- (blocking);
\draw[loop] (blocking) -- (solver);

% ============================================================
% 出力層
% ============================================================
\node[data, below=18mm of solver, xshift=-20mm] (solutions)
{解集合\\$\{v_1,v_2,\dots\}$};

\node[data, below=18mm of solver, xshift=+20mm] (stats)
{統計情報\\変数数 / 節数};

\draw[flow] (solver) -- (solutions);
\draw[flow] (solver) -- (stats);

% ============================================================
% レイヤ枠
% ============================================================
\begin{pgfonlayer}{background}
\node[draw=black!30, rounded corners, thick,
      fit=(dimacs)(graph),
      label=above:{入力層}] {};

\node[draw=black!30, rounded corners, thick,
      fit=(mds)(mis)(mds_basic)(mds_trans)(mis_basic)(mis_trans),
      label=above:{問題定式化層}] {};

\node[draw=black!30, rounded corners, thick,
      fit=(csp),
      label=above:{制約層(CSP)}] {};

\node[draw=black!30, rounded corners, thick,
      fit=(encoder),
      label=above:{エンコード層}] {};

\node[draw=black!30, rounded corners, thick,
      fit=(solver)(blocking),
      label=above:{SAT ソルバ層}] {};

\node[draw=black!30, rounded corners, thick,
      fit=(solutions)(stats),
      label=above:{出力層}] {};
\end{pgfonlayer}

\end{tikzpicture}
%
%   }
%   \caption{CSP--SAT に基づく極小・極大集合列挙アーキテクチャ}
%  \label{fig:archi}
% \end{figure}
