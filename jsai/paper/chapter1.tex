% %%%%%%%%%%%%%%%%%%%%%%%%%%%%%%%%%%%%%%%%%%%%%%%%%%%%%%%%%% 
% \chapter{はじめに} \label{chap:introduction}
% \pagenumbering{arabic}
% %%%%%%%%%%%%%%%%%%%%%%%%%%%%%%%%%%%%%%%%%%%%%%%%%%%%%%%%%% 

% モデルとは,命題変数からなる論理式に対して,各変数に真偽値を割り当てる
% ことにより,その論理式を真にする割り当てのことである.命題論理式の充足
% 可能性判定問題(SAT 問題)とは,このようなモデルが存在するかどうかを判
% 定する問題である.近年は SAT ソルバーの高性能化により,大規模かつ複雑
% な論理式に対しても実用的な時間でモデルを得られるようになっており基盤的手法として用いられている~\cite{%
% DBLP:series/faia/Biere09,%
% jackson2006softwareabstractions,%
% DBLP:conf/ecai/KautzS92,%
% CSC06:TamuraB08,%
% DBLP:journals/constraints/TamuraTKB09}.
% 一方で,実際の応用においては,単にモデルが存在するかどうかだけでなく,
% 得られるモデルの中からどのような代表的解を抽出するかが重要となる.

% 極小・極大モデルとは,命題論理式のモデル集合に包含関係を導入して定義さ
% れる,真の変数集合をこれ以上減らせない・増やせないモデルである.極小・
% 極大モデルは冗長性のない代表的なモデルと捉えることもでき,論理プログラ
% ミングの非単調推論や circumscription に基づく常識推論の意味論的基盤を
% 成すとともに,グラフ理論における極小支配集合や極大独立集合,多目的最適
% 化におけるパレートフロントの計算など,幅広い応用と対応してい
% る~\cite{niemela96,soh}.これらの応用において,極小・極大モデルを高速
% に計算することは重要な研究課題である.

% 既存の計算方法~\cite{koshimura2009minimal}としてSATソルバーを複数回起動して徐々にモデルを包含関係において大きく・小さくする方法が提案されてきたが,この方法ではSATソルバーの起動回数が増加し,計算コストが大きくなるという課題がある.

% 本論文では,SAT符号化を用いてSATソルバー1回起動のみで極小モデルおよび極大モデルを計算する方法に注目する.論文\cite{adachi2023}では,SAT問題に符号化を施したCNFに変換を加えることで,SATソルバーを1回起動するだけで極小モデルを計算する手法が提案されている.しかし,このようなSAT符号化に基づく手法については,極小モデルに関するものが中心であり,極大モデルに対する同様の手法や,それらの性能評価は十分に行われていなかった.そこで,SAT符号化を用いてSATソルバーを1回起動することで極大モデルを計算するためのCNF変換方法を提案する.また,SATソルバーの1回起動および複数回起動に基づく極小モデル計算・極大モデル計算手法の実装や,グラフ問題に基づくベンチマークを用いた実行実験による従来手法との性能比較を行った.SATソルバー1回起動による極小モデルの計算方法では,SAT問題に符号化を施したCNFの正リテラルに注目し,ある節中の他の変数で節が満たされているとき,その変数を偽にするという制約を追加することにより,極小モデル計算を実現した.また,SATソルバー1回起動による極大モデルの計算方法では,SAT問題に符号化を施したCNFの負リテラルに注目し,ある節中の他の変数で節が満たされているとき,その変数を真にするという制約を追加することにより,極大モデル計算を実現した.

% % 本論文の構成は以下のとおりである.第2章では,SAT 問題および命題論理式の基本的な定義を与え,極小モデルおよび極大モデルの概念について説明する.また,SAT ソルバーを複数回起動することによる従来の極小・極大モデル計算方法について述べる.第3章では,SAT 符号化を用いた極小モデル計算の既存手法を整理した上で,極大モデル計算のための CNF 変換方法を提案し,そのアルゴリズムおよび実装について説明する.第4章では,提案手法および既存手法に対して実行実験を行い,計算時間や解ける問題規模の観点から性能評価を行う.最後に,第5章では,本論文のまとめと今後の課題について述べる.



% %%% Local Variables:
% %%% mode: japanese-latex
% %%% TeX-master: "paper"
% %%% End:


\section{はじめに}
\label{chap:introduction}
%%\pagenumbering{arabic}

\textbf{モデル}とは,命題変数からなる論理式に対して,各変数に真偽値を割り当てることにより,その論理式を真にする割り当てのことである.

\textbf{命題論理式の充足可能性判定問題(SAT 問題)}とは,与えられた命題論理式に対して,このようなモデルが存在するかどうかを判定する問題である.
近年は SAT ソルバーの高性能化により,大規模かつ複雑な論理式に対しても実用的な時間でモデルを得られるようになっており,様々な分野で推論基盤として用いられている~\cite{DBLP:series/faia/Biere09,jackson2006softwareabstractions,DBLP:conf/ecai/KautzS92,CSC06:TamuraB08,DBLP:journals/constraints/TamuraTKB09}.
一方で,実際の応用においては,単にモデルが存在するかどうかだけでなく,得られるモデルの中からどのような代表的解を抽出するかが重要となる.

\textbf{極小・極大モデル}とは,命題論理式のモデル集合に包含関係を導入して定義される,真の変数集合をこれ以上減らせないモデル,および真の変数集合をこれ以上増やせないモデルである.
極小・極大モデルは冗長性のない代表的なモデルと捉えることもでき,論理プログラミングの非単調推論や circumscription に基づく常識推論の意味論的基盤を成すとともに,グラフ理論における極小支配集合や極大独立集合,多目的最適化におけるパレートフロントの計算など,幅広い応用と対応している~\cite{niemela96,soh}.
これらの応用において,極小・極大モデルを高速に計算することは重要な研究課題である.

既存の計算方法~\cite{koshimura2009minimal}として,SAT ソルバーを複数回起動して徐々にモデルを包含関係において大きく・小さくする方法で極小・極大モデルを計算する方法が提案された.
しかし,この方法ではモデル候補ごとに SAT ソルバーの起動が増加し,さらに極小性・極大性を保証するために充足不能性の判定を行う必要があるため,計算コストが大きくなるという課題がある.

そこで,SAT 符号化を用いて SAT ソルバーを1回起動するのみで極小モデルおよび極大モデルを計算する方法に注目する.
この方法に関する足立の先行研究では,与えられた連言標準形の命題論理式 (CNF式) $\Psi$ に適切な変換を加えた $f(\Psi)$ を得ることで,$\Psi$ の極小モデルが $f(\Psi)$のモデルと一致することが示された.
しかし,このような SAT 符号化に基づく手法については,極小モデルに関するものが中心であり,極大モデルに対する同様の手法や,それらの性能評価は十分に行われていなかった.

本論文では,SAT 符号化を用いて SAT ソルバーを1回起動することで極大モデルを計算するための CNF 変換方法を提案する.
また,SAT ソルバーの1回起動および複数回起動に基づく極小モデル計算・極大モデル計算手法の実装を行い,グラフ問題に基づくベンチマークを用いた実行実験による従来手法との性能比較を行った.
SAT ソルバー1回起動による極小モデルの計算方法では,SAT 問題に符号化を施した CNF の正リテラルに注目し,ある節中の他の変数で節が満たされているとき,その変数を偽にするという制約を追加することにより,極小モデル計算を実現した.
また,SAT ソルバー1回起動による極大モデルの計算方法では,SAT 問題に符号化を施した CNF の負リテラルに注目し,ある節中の他の変数で節が満たされているとき,その変数を真にするという制約を追加することにより,極大モデル計算を実現した.

本論文の主な貢献と結果は以下の通りである.
\begin{itemize}
    \item 極大モデルに対しても SAT 符号化を用いて SAT ソルバーを1回だけ起動することで計算する方法を提案した.足立~\cite{adachi2023}の極小モデルの変換では各正リテラルに対して制約を追加するのに対し,本手法では各負リテラルに対して制約の追加を行う.これにより先行研究と併せて極小・極大モデルの両方を計算するための基盤を構築した.
    \item 符号化を用いて極小・極大モデルを列挙する手法を SAT ソルバーと Rust 言語を用いて実装した.このプログラムは CNF 式を入力とし,提案する符号化を適用した上で SAT ソルバーを呼び出すことにより極小・極大モデルを列挙する.
    \item 提案手法の有効性を検証するために,極小支配集合問題 (MDS) と極大独立集合問題 (MIS) の計50問のベンチマーク問題を用いて評価を行った.既存手法と提案手法とで全ての極小・極大モデルを列挙する CPU 時間を比較した結果,提案手法は既存手法よりも多くの問題を高速に解くことに成功し,符号化を用いた方法の有効性を確認した.
\end{itemize}

提案手法は,SATソルバーの単一呼び出しで極小・極大モデルを計算可能にするものであり,組合せ問題などにおける解の計算技術に寄与するものである.