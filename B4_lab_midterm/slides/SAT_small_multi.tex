\begin{tikzpicture}[
  >=Stealth,
  every node/.style={transform shape},
  node distance=10mm and 22mm,
  box/.style={
    draw, rounded corners=8pt,
    minimum width=24mm,
    minimum height=7mm,
    align=center
  },
  lbl/.style={font=\footnotesize, align=center} % ← これを追加
]
  \node[box] (prob) {問題};
  \node[box, right=20mm of prob] (sat) {SAT問題};
  \node[box, below=20mm of sat] (mmin) {SAT問題の\\モデル};
  \node[box, left=20mm of mmin] (ans) {問題の\\解};

  \draw[->] (prob.east) -- node[midway, above, lbl]{SAT符号化} (sat.west);
  \draw[->] (sat.south) -- node[midway, left, lbl]{SATソルバー\\一回起動} (mmin.north); % ← OK になる
  \draw[->] (mmin.west) -- node[midway, above, lbl]{復号化} (ans.east);
  \draw[->, dashed] (prob.south) -- (ans.north);
\end{tikzpicture}
