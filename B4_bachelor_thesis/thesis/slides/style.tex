\definecolor{teal}{rgb}{0.0,0.6,0.6}
\definecolor{bluebluecyan}{rgb}{0,.6,0.8}
\definecolor{colortitle}{rgb}{0.0,0.5,0.5}

\colorlet{couleurtheme}{teal}  % Couleur principale du thème
\colorlet{couleurcit}{gray}  % Couleur des citations
\colorlet{couleurex}{gray}  % Couleur des exemples
\colorlet{couleurredex}{red}  % Couleur des exemples
\colorlet{couleurliens}{darkblue}  % Couleur des liens

% \usetheme{Pittsburgh}   % Thème général
% \usefonttheme{default}  % Thème de polices
% \setbeamertemplate{navigation symbols}{}  % Pas de menu de navigation
%\setbeamertemplate{itemize item}[x]   % Puces des listes

% \usecolortheme[named=couleurtheme]{structure}    % Couleur de la structure : titres et puces
%\setbeamercolor{normal text}{bg=black,fg=white}  % Couleur du texte
% \setbeamercolor{item}{fg=couleurtheme}           % Couleur des puces
%\setbeamercolor{item projected}{fg=black}        % Couleur des recouvrements
%\setbeamercolor{alerted text}{fg=yellow}         % ?
% \setbeamerfont{frametitle}{size=\Large}  % Police des titres


% Flèche grise
\newcommand{\f}{\textcolor{couleurtheme}{\textbf{$\rightarrow$\ }}}
\newcommand{\F}{\textcolor{couleurtheme}{\textbf{$\Rightarrow$\ }}}

% Environnement liste avec flèches
\newenvironment{fleches}{%
  \begin{list}{}{%
      \setlength{\labelwidth}{1em}% largeur de la boîte englobant le label
      \setlength{\labelsep}{0pt}% espace entre paragraphe et l’étiquette
      %\setlength{\itemsep}{1pt}
      %\setlength{\leftmargin}{\labelwidth+\labelsep}% marge de gauche
      \renewcommand{\makelabel}{\f}%
    }}{\end{list}}

% Liste sans puce
\newenvironment{liste}{%
  \begin{list}{}{%
      \setlength{\labelwidth}{0em}% largeur de la boîte englobant le label
      \setlength{\labelsep}{0pt}% espace entre paragraphe et l’étiquette
      \setlength{\leftmargin}{0em}% marge de gauche
      %\renewcommand{\makelabel}{\f}%
    }}{\end{list}}

% Style des exemples
\newcommand{\ex}[1]{\textcolor{couleurex}{#1}}
\newcommand{\qex}[1]{\quad \ex{#1}}
\newcommand{\rex}[1]{\hfill \ex{#1}}
\newcommand{\redex}[1]{\textcolor{couleurredex}{#1}}

\newcommand{\lien}[1]{\textcolor{couleurliens}{\underline{\url{#1}}}}

%\newcommand{\console}[1]{\textcolor{darkgray}{#1}}

% Style des citations
\newcommand{\tscite}[1]{\textcolor{couleurcit}{#1}}
\newcommand{\tcite}[1]{\tscite{[#1]}}
\newcommand{\tcitebullet}{~~$\textcolor{couleurtheme}{\bullet}$~}



% Style de texte mis en valeur
\newcommand{\tval}[1]{\textbf{#1}}

% Un vrai symbole pour l'ensemble vide
\renewcommand{\emptyset}{\varnothing}

% Pour définir la conférence et son nom court
%\newcommand{\conference}[2]{\def\theconference{#2}
% \def\insertshortconference{\ifthenelse{\equal{#1}{-}}{#2}{\ifthenelse{\equal{#1}{}}{#2}{#1}}}}

% \setbeamertemplate{footline}{\color{couleurtheme}%
% \scriptsize
% \quad\strut%
% \insertauthor%
% \hfill%
% \insertframenumber/\inserttotalframenumber%
% \hfill%
% \insertshortconference{} --- \thedate\quad\strut
% }


\newcommand{\headersep}{$\circ$} % \bullet \triangleright

% \setbeamertemplate{headline}{\color{couleurtheme}%
%   \vskip0.3em%
%   \quad\strut%
%   {\scriptsize\color{black}%
%     % Gris si une section existe
%     \ifthenelse{\equal{\thesection}{0}}{}{%
%       \ifthenelse{\equal{\lastsection}{x}}{}{%
%         \color{couleurtheme}%
%       }}%
%     \insertshorttitle
%     \ifthenelse{\equal{\thesection}{0}}{}{%
%       \ifthenelse{\equal{\lastsection}{x}}{}{%
%         ~\headersep{} %
%         % Gris si une sous-section existe
%         \ifthenelse{\equal{\thesubsection}{0}}{\color{black}}{%
%           \ifthenelse{\equal{\lastsubsection}{x}}{\color{black}}{%
%             \color{couleurtheme}%
%           }}%
%         \insertsectionhead%
%         %
%         \ifthenelse{\equal{\thesubsection}{0}}{}{%
%           \ifthenelse{\equal{\lastsubsection}{x}}{}{%
%             ~\headersep{} \color{black}\insertsubsectionhead%
%             %
%           }}}}}%
%   \vskip-5ex%
% }

\def \scaleex {0.85}
\def \scaleminiex {0.6}
\def \scaleinf {0.6}

\colorlet{colorb}{blue}
\colorlet{colora1}{yellow}
\colorlet{colora0}{green}
\colorlet{colora1font}{darkyellow}
\colorlet{colora0font}{darkgreen}

\colorlet{exanswer}{blue}
\colorlet{colorgray}{lightgray}

%\definecolor{colortitle}{rgb}{1,0.854,0.725}
%\definecolor{colortitle}{rgb}{0.333,0.101,0.545}
\definecolor{colortitle}{rgb}{0,0.6,0.6}

%255-218-185
%85;26;139
%104;34;139
%\colorlet{colortitle}{bluebluecyan!50}

%\setbeamertemplate{blocks}[default]