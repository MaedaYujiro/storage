%%%%%%%%%%%%%%%%%%%%%%%%%%%%%%%%%%%%%%%%%%%%%%%%%%%%%%%%%% 
\chapter{はじめに} \label{chap:introduction}
\pagenumbering{arabic}
%%%%%%%%%%%%%%%%%%%%%%%%%%%%%%%%%%%%%%%%%%%%%%%%%%%%%%%%%% 

モデルとは,命題変数からなる論理式を真にする値割当てのことである.命題
論理式の充足可能性判定問題(SAT 問題)とは,このようなモデルが存在する
かどうかを判定する問題である.近年は SAT ソルバーの高性能化により,大
規模かつ複雑な論理式に対しても実用的な時間でモデルを得られるようになっ
ており基盤的手法として用いられている~\cite{%
  DBLP:series/faia/Biere09,% jackson2006softwareabstractions,%
  DBLP:conf/ecai/KautzS92,% CSC06:TamuraB08,%
  DBLP:journals/constraints/TamuraTKB09}.一方で,実際の応用においては,
単にモデルが存在するかどうかだけでなく,得られるモデルの中からどのよう
な代表的解を抽出するかが重要となる.

極小・極大モデルとは,命題論理式のモデル集合に包含関係を導入して定義さ
れる,真に値割当てされる変数の集合をこれ以上減らせない・増やせないモデ
ルである.極小・極大モデルは冗長性のない代表的なモデルと捉えることもで
き,論理プログラミングの非単調推論や circumscription に基づく常識推論
の意味論的基盤を成すとともに,グラフ理論における極小支配集合や極大独立
集合,多目的最適化におけるパレートフロントの計算など,幅広い応用と対応
している~\cite{niemela96,soh}.これらの応用において,極小・極大モデル
を高速に計算することは重要な研究課題である.

既存の計算方法~\cite{koshimura2009minimal}としてSATソルバーを複数回起動して徐々にモデルを包含関係において大きく・小さくする方法が提案されてきたが,この方法ではSATソルバーの起動回数が増加し,計算コストが大きくなるという課題がある.

本論文では,SAT符号化を用いてSATソルバー1回起動のみで極小モデルおよび
極大モデルを計算する方法に注目し,その実装と評価を行う.先行研
究~\cite{adachi2023}では,与えられた連言標準形の命題論理式 (CNF式)
$\Psi$ に適切な変換を加えた $f(\Psi)$ を得ることで,$\Psi$ の極小モデ
ルを $f(\Psi)$ のモデルと一致することが示された.本研究では,まず極小
モデルだけではなく極大モデルに対しても同様の変換を実現できることを示し
た.具体的には,与えられたCNF式の各節に含まれる各負リテラルについて,
(i) その負リテラル $\ell$ を含む節の集合から,その負リテラルを除いた節の連言 $\psi_{\ell}$ を構築する,
(ii) 制約 $\psi_{\ell} \leftarrow \ell$ を $\Psi$ に追加する,
という手順で変換を行う.
この追加制約により,$\ell$ の真偽が全体の充足に影響しない場合には $\ell$ が真に値割当てされる.

次に,実際に与えられたCNF式の極小・極大モデルを計算するプログラムを実装し,
評価を行った.評価ベンチマークには,独立集合問題と支配集合問題を表現したCNF式を用いた.
全ての極小・極大モデルを列挙するCPU時間を提案方法とSATソルバーを複数回起動する既存の計算方法で比較した.
結果として,提案方法は全列挙を完了するのに必要なSAT呼び出し回数が少なく高速であることが示された.



%%% Local Variables:
%%% mode: japanese-latex
%%% TeX-master: "paper"
%%% End:
