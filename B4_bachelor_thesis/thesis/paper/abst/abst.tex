\documentclass{abst}

%%% Packages
\usepackage[dvipdfmx]{graphicx}
\usepackage{lineno}
%\usepackage{bm}
\usepackage{array}
\usepackage{url}
\usepackage{alltt}
\usepackage{ascmac}
\usepackage{tikz}
\usepackage{subcaption}
\usepackage{longtable}
\usepackage{seqsplit}
\usepackage{appendix}
\usepackage{comment}
\usetikzlibrary{arrows,shapes}
\usetikzlibrary{positioning}
\usepackage{listings}
\usepackage{plistings}

%%% 数学パッケージ
\usepackage{amsmath}
\usepackage{amssymb}
\usepackage{mathtools}

%%% 表組み関連パッケージ
\usepackage{booktabs}
\usepackage{tabularx}
\usepackage{float}
\usepackage{placeins}

%%% アルゴリズム関連パッケージ
\usepackage{algorithm}
\usepackage{algpseudocode}

%%% 定理環境の定義
\usepackage{amsthm}
\newtheorem{theorem}{定理}
\newtheorem{lemma}[theorem]{補題}
\newtheorem{proposition}[theorem]{命題}
\newtheorem{corollary}[theorem]{系}
\theoremstyle{definition}
\newtheorem{definition}{定義}
\newtheorem{example}{例}
\newtheorem{formula}{式}
\theoremstyle{remark}
\newtheorem{remark}{注意}

\def\lstlistingname{コード}
\def\lstlistlistingname{コード目次}
%%\renewcommand{\bibname}{参考文献}
\captionsetup[figure]{labelformat=default}
\usepackage{color}

%%% For ASP
\newcommand{\asap}{\textit{teaspoon}}
\newcommand{\gringo}{\textit{gringo}}
\newcommand{\clingo}{\textit{clingo}}
\newcommand{\dlv}{\textit{DLV}}
\newcommand{\wasp}{\textit{WASP}}
\newcommand{\code}[1]{\lstinline[basicstyle=\ttfamily]{#1}}
\newcommand{\naf}[1]{\ensuremath{{\sim\!\!{#1}}}}
\newcommand{\head}[1]{\ensuremath{\mathit{head}(#1)}}
\newcommand{\body}[1]{\ensuremath{\mathit{body}(#1)}}
%\newcommand{\atom}[1]{\ensuremath{\mathit{atom}(#1)}}
\newcommand{\poslits}[1]{\ensuremath{{#1}^+}}
\newcommand{\neglits}[1]{\ensuremath{{#1}^-}}
\newcommand{\pbody}[1]{\poslits{\body{#1}}}
\newcommand{\nbody}[1]{\neglits{\body{#1}}}
%\newcommand{\Cn}[1]{\ensuremath{\mathit{Cn}(#1)}}
\newcommand{\reduct}[2]{\ensuremath{#1^{#2}}}

%%% Local Variables:
%%% mode: japanese-latex
%%% TeX-master: "paper"
%%% End:

\begin{document}

%%%%%%%%%%%%%%%%%%%%%%%%%%%%%%%%%%%%%%%%%%%%%%%%%%%%%%%%%%%%%%%%%%%
\研究室名{番原・宋}
\氏名{~~前~~田~~悠~~士~~朗~~}
\卒論題目{%
SAT符号化を用いた命題論理式の極小・極大モデル計算の実装と評価
}

%%%%%%%%%%%%%%%%%%%%%%%%%%%%%%%%%%%%%%%%%%%%%%%%%%%%%%%%%%%%%%%%%%%
\卒論要旨{%
%

\textbf{モデル}とは,命題論理式を真にする値割当てである.充足可能性判
定問題(SAT 問題)とは,与えられた命題論理式のモデルの存在を判定する
問題である.SATソルバーはSAT問題を解くプログラムであり,存在する場合に
はモデルを出力する.近年は SAT ソルバーの高性能化により,大規模かつ複
雑な論理式に対しても実用的な時間でモデルを得られるようになっており,様々
な分野で推論基盤として用いられている.一方で,実際の応用においては,単
にモデルを出力・列挙するだけでなく,得られるモデルの中からどのような代
表的モデルを抽出するかが重要となる.

\textbf{極小・極大モデル}とは,命題論理式のモデル集合に包含関係を導入
したとき,真に値割当てされる変数の集合をこれ以上減らせないモデル,およ
び増やせないモデルをそれぞれ指す.論理プログラミングにおける非単調推論
の基盤であるとともに,極小支配集合,極大独立集合などのグラフ理論の問題,
多目的最適化問題におけるパレートフロント計算など幅広い応用を持つ.これ
らの応用において,極小・極大モデルを高速に計算することは重要な研究課題
である.しかし,既存の計算方法では,SATソルバーを複数回起動してモデル
を計算し,さらに極小性・極大性を保証するために充足不能性の判定を行う必
要があり,計算コストが大きいという課題がある.

本論文では,SAT符号化を用いてSATソルバーを1回だけ起動することで極小モ
デルおよび極大モデルを計算する方法を提案する.この方法に関する足立の先
行研究では,与えられた連言標準形の命題論理式 (CNF式) $\Psi$ に適切な変
換を加えた $f(\Psi)$ を得ることで,$\Psi$ の極小モデルが $f(\Psi)$のモ
デルと一致することが示された.本研究では,極小モデルだけではなく極大モ
デルに対しても同様の変換を実現できることを示し,この変換を用いた極小・
極大モデルの計算方法を実装・評価する.

本論文の主な貢献と結果は以下の通りである.
\begin{itemize}
\item 極大モデルに対してもSAT符号化を用いてSATソルバーを1回だけ起動することで計算する方法を提案した.
  足立の極小モデルの変換では各正リテラルに対して制約を追加するのに対し,極大モデルでは各負リテラルに対して制約の追加を行う.
  これにより先行研究と併せて極小・極大モデルの両方を計算するための基盤が整った.
\item 符号化を用いて極小・極大モデルを列挙する手法をSATソルバーと Rust 言語を用いて実装した.
  このプログラムはCNF式を入力とし,提案する符号化を適用した上でSATソルバーを呼び出すことにより極小・極大モデルを列挙する.
  提案手法は充足不能性の判定を行う必要がなく,インクリメンタルSATを用いた効果的な列挙を期待できる.
\item 提案手法の有効性を検証するために,極大独立集合問題と極小支配集合問題の計32問のベンチマーク問題を用いて評価を行った.
既存手法と提案手法とで全ての極小・極大モデルを列挙するCPU時間を比較した.
結果として,既存手法の23問に対し,提案手法は30問をより高速に解くことに成功し,符号化を用いた方法の有効性を確認することができた.
\end{itemize}

提案手法は,SATソルバーの単一呼び出しで極小・極大モデルを計算可能にするものであり,組合せ問題などにおける解の計算技術に寄与するものである.
}

\end{document}
