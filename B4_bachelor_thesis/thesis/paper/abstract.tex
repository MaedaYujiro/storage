%%%%%%%%%%%%%%%%%%%%%%%%%%%%%%%%%%%%%%%%%%%%%%%%%%%%%%%%%% 
\chapter*{概要}
\pagenumbering{roman}
%%%%%%%%%%%%%%%%%%%%%%%%%%%%%%%%%%%%%%%%%%%%%%%%%%%%%%%%%% 

本論文では,SAT符号化を用いた命題論理式の極小・極大モデルの効率的な計算方法について述べる.
極小モデルとは,CNFで与えられる命題論理式のモデル集合に包含関係を入れて定義され,真となる変数をこれ以上減らせないモデルである.また,極大モデルとは,真となる変数をこれ以上増やせないモデルである.極小モデルや極大モデルを計算することは整数ナップサック問題の圧縮解計算や最適化問題の多目的パレートフロント求解において有用である.従来の反復起動法はモデル候補ごとにSATソルバー起動が増え計算コストが大きい.
本研究では,負リテラルに着目し他リテラルで節が満たされるなら当該変数を真にする制約を追加することで,SATソルバー1回の起動で極大モデルを計算できる極大変換を提案した.また,既存のSATソルバー1回の起動による極小モデル計算と提案する極大モデル計算を実装し,$3 \times n$のグリッドグラフを用いて評価実験を行った.
その結果,SATソルバーの起動回数が大幅に削減され,反復起動法よりも多くの問題を制限時間内に解けることを確認した.

