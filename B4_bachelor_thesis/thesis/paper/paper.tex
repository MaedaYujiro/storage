
\PassOptionsToPackage{dvipdfmx}{graphicx} 

%%% for platex
\documentclass[dvipdfmx,a4paper,12pt]{jbook}
%%% for lualatex
%\documentclass[a4paper,12pt]{ltjbook}

\usepackage{bachelor}
\usepackage{comment}
\usepackage{afterpage}
\usepackage{mathtools}
\usepackage{booktabs}
\usepackage{amsmath,amssymb}
\usepackage{float} 
\usepackage{placeins}
\usepackage{algorithm}
\usepackage{algpseudocode}
 \makeatletter
 \let\Return\@undefined
 \makeatother
 \algrenewcommand\algorithmicrequire{\textbf{Input:}}
 \algrenewcommand\algorithmicensure{\textbf{Output:}}
\usepackage{tabularx}
\usepackage{array}
\usepackage{tikz}
\usetikzlibrary{
  arrows.meta,
  positioning,
  shapes.multipart,
  fit,
  backgrounds,
  calc
}
\usepackage{caption}
\usepackage{titlesec}
\titleformat{\paragraph}[block]
  {\normalfont\large\bfseries}
  {}
  {0pt}
  {}
\titlespacing*{\paragraph}{0pt}{3.25ex plus 1ex minus .2ex}{1ex}



\newtheorem{definition}{定義}
\newtheorem{example}{例}
\newtheorem{formula}{式}

%%% Packages
\usepackage[dvipdfmx]{graphicx}
\usepackage{lineno}
%\usepackage{bm}
\usepackage{array}
\usepackage{url}
\usepackage{alltt}
\usepackage{ascmac}
\usepackage{tikz}
\usepackage{subcaption}
\usepackage{longtable}
\usepackage{seqsplit}
\usepackage{appendix}
\usepackage{comment}
\usetikzlibrary{arrows,shapes}
\usetikzlibrary{positioning}
\usepackage{listings}
\usepackage{plistings}

%%% 数学パッケージ
\usepackage{amsmath}
\usepackage{amssymb}
\usepackage{mathtools}

%%% 表組み関連パッケージ
\usepackage{booktabs}
\usepackage{tabularx}
\usepackage{float}
\usepackage{placeins}

%%% アルゴリズム関連パッケージ
\usepackage{algorithm}
\usepackage{algpseudocode}

%%% 定理環境の定義
\usepackage{amsthm}
\newtheorem{theorem}{定理}
\newtheorem{lemma}[theorem]{補題}
\newtheorem{proposition}[theorem]{命題}
\newtheorem{corollary}[theorem]{系}
\theoremstyle{definition}
\newtheorem{definition}{定義}
\newtheorem{example}{例}
\newtheorem{formula}{式}
\theoremstyle{remark}
\newtheorem{remark}{注意}

\def\lstlistingname{コード}
\def\lstlistlistingname{コード目次}
%%\renewcommand{\bibname}{参考文献}
\captionsetup[figure]{labelformat=default}
\usepackage{color}

%%% For ASP
\newcommand{\asap}{\textit{teaspoon}}
\newcommand{\gringo}{\textit{gringo}}
\newcommand{\clingo}{\textit{clingo}}
\newcommand{\dlv}{\textit{DLV}}
\newcommand{\wasp}{\textit{WASP}}
\newcommand{\code}[1]{\lstinline[basicstyle=\ttfamily]{#1}}
\newcommand{\naf}[1]{\ensuremath{{\sim\!\!{#1}}}}
\newcommand{\head}[1]{\ensuremath{\mathit{head}(#1)}}
\newcommand{\body}[1]{\ensuremath{\mathit{body}(#1)}}
%\newcommand{\atom}[1]{\ensuremath{\mathit{atom}(#1)}}
\newcommand{\poslits}[1]{\ensuremath{{#1}^+}}
\newcommand{\neglits}[1]{\ensuremath{{#1}^-}}
\newcommand{\pbody}[1]{\poslits{\body{#1}}}
\newcommand{\nbody}[1]{\neglits{\body{#1}}}
%\newcommand{\Cn}[1]{\ensuremath{\mathit{Cn}(#1)}}
\newcommand{\reduct}[2]{\ensuremath{#1^{#2}}}

%%% Local Variables:
%%% mode: japanese-latex
%%% TeX-master: "paper"
%%% End:
 % 自分用のマクロ

%%%%%%%%%%%%%%%%%%%%%%%%%%%%%%%%%%%%%%%%%%%%%%%%%%%%%%%%%%
% タイトル
%%%%%%%%%%%%%%%%%%%%%%%%%%%%%%%%%%%%%%%%%%%%%%%%%%%%%%%%%%
%\school{名古屋大学情報学部\\} 
\bookname{コンピュータ科学科\\情報システム系\\卒業論文}
\title{\parbox{1.05\textwidth}{\centering
SAT符号化を用いた命題論理式の\\極小・極大モデル計算の実装と評価}}
\date{2026年2月}
\author{102230349~~~前田 悠士朗}
%%%%%%%%%%%%%%%%%%%%%%%%%%%%%%%%%%%%%%%%%%%%%%%%%%%%%%%%%% 
% 本体
%%%%%%%%%%%%%%%%%%%%%%%%%%%%%%%%%%%%%%%%%%%%%%%%%%%%%%%%%% 
\begin{document}
\maketitle

%%%%%%%%%%%%%%%%%%%%%%%%%%%%%%%%%%%%%%%%%%%%%%%%%%%%%%%%%% 
\section*{概要}
%%\pagenumbering{roman}
%%%%%%%%%%%%%%%%%%%%%%%%%%%%%%%%%%%%%%%%%%%%%%%%%%%%%%%%%% 

命題論理式の極小モデル・極大モデルとは,モデル集合に包含関係を導入し,真の変数集合をこれ以上減らせない・増やせないモデルである.従来の反復起動法はモデル候補ごとにSATソルバーの起動が増え計算コストが大きい.本研究では,CNFの各リテラルに着目し,他リテラルで節が充足される場合に当該変数の真偽を定める制約を追加することで,SATソルバー1回の起動で1つの極小・極大モデルを得るCNF変換を提案・実装した.$3 \times n$のグリッドグラフに基づく極小支配集合・極大独立集合問題で評価した結果,提案手法はSATソルバーの起動回数を大幅に削減し,反復起動法より多くの問題を制限時間内に解けることを確認した.

    % 概要

\tableofcontents    % 目次
\listoffigures      % 図の目次
\listoftables       % 表の目次
%\lstlistoflistings  % コードの目次

% ここから「本文」
% %%%%%%%%%%%%%%%%%%%%%%%%%%%%%%%%%%%%%%%%%%%%%%%%%%%%%%%%%% 
% \chapter{はじめに} \label{chap:introduction}
% \pagenumbering{arabic}
% %%%%%%%%%%%%%%%%%%%%%%%%%%%%%%%%%%%%%%%%%%%%%%%%%%%%%%%%%% 

% モデルとは,命題変数からなる論理式に対して,各変数に真偽値を割り当てる
% ことにより,その論理式を真にする割り当てのことである.命題論理式の充足
% 可能性判定問題(SAT 問題)とは,このようなモデルが存在するかどうかを判
% 定する問題である.近年は SAT ソルバーの高性能化により,大規模かつ複雑
% な論理式に対しても実用的な時間でモデルを得られるようになっており基盤的手法として用いられている~\cite{%
% DBLP:series/faia/Biere09,%
% jackson2006softwareabstractions,%
% DBLP:conf/ecai/KautzS92,%
% CSC06:TamuraB08,%
% DBLP:journals/constraints/TamuraTKB09}.
% 一方で,実際の応用においては,単にモデルが存在するかどうかだけでなく,
% 得られるモデルの中からどのような代表的解を抽出するかが重要となる.

% 極小・極大モデルとは,命題論理式のモデル集合に包含関係を導入して定義さ
% れる,真の変数集合をこれ以上減らせない・増やせないモデルである.極小・
% 極大モデルは冗長性のない代表的なモデルと捉えることもでき,論理プログラ
% ミングの非単調推論や circumscription に基づく常識推論の意味論的基盤を
% 成すとともに,グラフ理論における極小支配集合や極大独立集合,多目的最適
% 化におけるパレートフロントの計算など,幅広い応用と対応してい
% る~\cite{niemela96,soh}.これらの応用において,極小・極大モデルを高速
% に計算することは重要な研究課題である.

% 既存の計算方法~\cite{koshimura2009minimal}としてSATソルバーを複数回起動して徐々にモデルを包含関係において大きく・小さくする方法が提案されてきたが,この方法ではSATソルバーの起動回数が増加し,計算コストが大きくなるという課題がある.

% 本論文では,SAT符号化を用いてSATソルバー1回起動のみで極小モデルおよび極大モデルを計算する方法に注目する.論文\cite{adachi2023}では,SAT問題に符号化を施したCNFに変換を加えることで,SATソルバーを1回起動するだけで極小モデルを計算する手法が提案されている.しかし,このようなSAT符号化に基づく手法については,極小モデルに関するものが中心であり,極大モデルに対する同様の手法や,それらの性能評価は十分に行われていなかった.そこで,SAT符号化を用いてSATソルバーを1回起動することで極大モデルを計算するためのCNF変換方法を提案する.また,SATソルバーの1回起動および複数回起動に基づく極小モデル計算・極大モデル計算手法の実装や,グラフ問題に基づくベンチマークを用いた実行実験による従来手法との性能比較を行った.SATソルバー1回起動による極小モデルの計算方法では,SAT問題に符号化を施したCNFの正リテラルに注目し,ある節中の他の変数で節が満たされているとき,その変数を偽にするという制約を追加することにより,極小モデル計算を実現した.また,SATソルバー1回起動による極大モデルの計算方法では,SAT問題に符号化を施したCNFの負リテラルに注目し,ある節中の他の変数で節が満たされているとき,その変数を真にするという制約を追加することにより,極大モデル計算を実現した.

% % 本論文の構成は以下のとおりである.第2章では,SAT 問題および命題論理式の基本的な定義を与え,極小モデルおよび極大モデルの概念について説明する.また,SAT ソルバーを複数回起動することによる従来の極小・極大モデル計算方法について述べる.第3章では,SAT 符号化を用いた極小モデル計算の既存手法を整理した上で,極大モデル計算のための CNF 変換方法を提案し,そのアルゴリズムおよび実装について説明する.第4章では,提案手法および既存手法に対して実行実験を行い,計算時間や解ける問題規模の観点から性能評価を行う.最後に,第5章では,本論文のまとめと今後の課題について述べる.



% %%% Local Variables:
% %%% mode: japanese-latex
% %%% TeX-master: "paper"
% %%% End:


\section{はじめに}
\label{chap:introduction}
%%\pagenumbering{arabic}

\textbf{モデル}とは,命題変数からなる論理式に対して,各変数に真偽値を割り当てることにより,その論理式を真にする割り当てのことである.

\textbf{命題論理式の充足可能性判定問題(SAT 問題)}とは,与えられた命題論理式に対して,このようなモデルが存在するかどうかを判定する問題である.
近年は SAT ソルバーの高性能化により,大規模かつ複雑な論理式に対しても実用的な時間でモデルを得られるようになっており,様々な分野で推論基盤として用いられている~\cite{DBLP:series/faia/Biere09,jackson2006softwareabstractions,DBLP:conf/ecai/KautzS92,CSC06:TamuraB08,DBLP:journals/constraints/TamuraTKB09}.
一方で,実際の応用においては,単にモデルが存在するかどうかだけでなく,得られるモデルの中からどのような代表的解を抽出するかが重要となる.

\textbf{極小・極大モデル}とは,命題論理式のモデル集合に包含関係を導入して定義される,真の変数集合をこれ以上減らせないモデル,および真の変数集合をこれ以上増やせないモデルである.
極小・極大モデルは冗長性のない代表的なモデルと捉えることもでき,論理プログラミングの非単調推論や circumscription に基づく常識推論の意味論的基盤を成すとともに,グラフ理論における極小支配集合や極大独立集合,多目的最適化におけるパレートフロントの計算など,幅広い応用と対応している~\cite{niemela96,soh}.
これらの応用において,極小・極大モデルを高速に計算することは重要な研究課題である.

既存の計算方法~\cite{koshimura2009minimal}として,SAT ソルバーを複数回起動して徐々にモデルを包含関係において大きく・小さくする方法で極小・極大モデルを計算する方法が提案された.
しかし,この方法ではモデル候補ごとに SAT ソルバーの起動が増加し,さらに極小性・極大性を保証するために充足不能性の判定を行う必要があるため,計算コストが大きくなるという課題がある.

そこで,SAT 符号化を用いて SAT ソルバーを1回起動するのみで極小モデルおよび極大モデルを計算する方法に注目する.
この方法に関する足立の先行研究では,与えられた連言標準形の命題論理式 (CNF式) $\Psi$ に適切な変換を加えた $f(\Psi)$ を得ることで,$\Psi$ の極小モデルが $f(\Psi)$のモデルと一致することが示された.
しかし,このような SAT 符号化に基づく手法については,極小モデルに関するものが中心であり,極大モデルに対する同様の手法や,それらの性能評価は十分に行われていなかった.

本論文では,SAT 符号化を用いて SAT ソルバーを1回起動することで極大モデルを計算するための CNF 変換方法を提案する.
また,SAT ソルバーの1回起動および複数回起動に基づく極小モデル計算・極大モデル計算手法の実装を行い,グラフ問題に基づくベンチマークを用いた実行実験による従来手法との性能比較を行った.
SAT ソルバー1回起動による極小モデルの計算方法では,SAT 問題に符号化を施した CNF の正リテラルに注目し,ある節中の他の変数で節が満たされているとき,その変数を偽にするという制約を追加することにより,極小モデル計算を実現した.
また,SAT ソルバー1回起動による極大モデルの計算方法では,SAT 問題に符号化を施した CNF の負リテラルに注目し,ある節中の他の変数で節が満たされているとき,その変数を真にするという制約を追加することにより,極大モデル計算を実現した.

本論文の主な貢献と結果は以下の通りである.
\begin{itemize}
    \item 極大モデルに対しても SAT 符号化を用いて SAT ソルバーを1回だけ起動することで計算する方法を提案した.足立~\cite{adachi2023}の極小モデルの変換では各正リテラルに対して制約を追加するのに対し,本手法では各負リテラルに対して制約の追加を行う.これにより先行研究と併せて極小・極大モデルの両方を計算するための基盤を構築した.
    \item 符号化を用いて極小・極大モデルを列挙する手法を SAT ソルバーと Rust 言語を用いて実装した.このプログラムは CNF 式を入力とし,提案する符号化を適用した上で SAT ソルバーを呼び出すことにより極小・極大モデルを列挙する.
    \item 提案手法の有効性を検証するために,極小支配集合問題 (MDS) と極大独立集合問題 (MIS) の計50問のベンチマーク問題を用いて評価を行った.既存手法と提案手法とで全ての極小・極大モデルを列挙する CPU 時間を比較した結果,提案手法は既存手法よりも多くの問題を高速に解くことに成功し,符号化を用いた方法の有効性を確認した.
\end{itemize}

提案手法は,SATソルバーの単一呼び出しで極小・極大モデルを計算可能にするものであり,組合せ問題などにおける解の計算技術に寄与するものである.
%%%%%%%%%%%%%%%%%%%%%%%%%%%%%%%%%%%%%%%%%%%%%%%%%%%%%%%%%% 
\section{命題論理式の極小・極大モデル} \label{chap:maxmin}
%%%%%%%%%%%%%%%%%%%%%%%%%%%%%%%%%%%%%%%%%%%%%%%%%%%%%%%%%% 
\section{SAT問題の定義}
\label{sec:defSAT}
本節では,論文\cite{井上克巳2010sat}をもとにSAT問題を定義する.\\
\textgt{命題論理式}の\textbf{充足可能性判定問題(SAT問題; Boolean Satisfiability Testing Problem)}とは,与えられた命題論理式を充足するような命題変数への値の割り当てが存在するかどうかを判定する問題である.

\begin{definition}[命題変数]\rm
  \textgt{命題変数}は,1または0の値をとる変数であり,それぞれ\textbf{真(true)},\textbf{偽(false)}を表す.
\end{definition}

\begin{definition}[命題論理式]\rm
命題変数の集合を $V$ とする.\textgt{命題論理式}の集合 $\mathrm{Form}(V)$ を次の生成規則により再帰的に定義する.
\begin{itemize}
 \item (原子式)任意の $x \in V$ は命題論理式である(すなわち $x \in \mathrm{Form}(V)$).
 \item (否定)$\psi \in \mathrm{Form}(V)$ ならば,$\neg \psi \in \mathrm{Form}(V)$.
 \item (二項結合)$\psi_1,\psi_2 \in \mathrm{Form}(V)$ ならば,
       $(\psi_1 \land \psi_2)$,$(\psi_1 \lor \psi_2)$,$(\psi_1 \to \psi_2)$ は命題論理式である.
\end{itemize}
ここで用いる論理結合子は表\ref{tab:logconnect}に示す4種類である.
\begin{table}[h]
 \centering
 \caption{論理結合子}
 \label{tab:logconnect}
 \begin{tabular}{c l l}
 \hline
 記号 & 意味 & \quad 英語 \\ \hline
 \(\land\) & かつ,連言 & and, conjunction \\
 \(\lor\)  & または,選言 & or, disjunction \\
 \(\to\) & ならば,含意 & implies, implication \\
 \(\neg\) & でない,否定 & not, negation \\
 \hline
 \end{tabular}
\end{table}
\end{definition}

\begin{definition}[充足]\rm
$V$ に対する真偽値割り当て(解釈)を $I:V \rightarrow \{1,0\}$ とする.任意の命題論理式 $\psi \in \mathrm{Form}(V)$ に対し,$I$ が $\psi$ を充足すること($I \models \psi$)を次のように再帰的に定義する.また,$I \models \psi$ が成り立たないことを $I \not\models \psi$ と表す.
\begin{itemize}
 \item (原子式)$I \models x \Leftrightarrow I(x)=1$ \quad($x \in V$)
 \item (否定)$I \models \neg \psi \Leftrightarrow I \not\models \psi$
 \item (連言)$I \models (\psi_1 \land \psi_2) \Leftrightarrow (I \models \psi_1 \ \text{かつ}\ I \models \psi_2)$
 \item (選言)$I \models (\psi_1 \lor \psi_2) \Leftrightarrow (I \models \psi_1 \ \text{または}\ I \models \psi_2)$
 \item (含意)$I \models (\psi_1 \to \psi_2) \Leftrightarrow (I \not\models \psi_1 \ \text{または}\ I \models \psi_2)$
\end{itemize}
\end{definition}

% \begin{definition}[命題論理式]\rm
%  \textgt{命題論理式}は,命題変数に対して,\textbf{論理結合子(operator)}を再帰的に適用したものである. \\
%  論理結合子には,表\ref{tab:logconnect}に示す4種類がある.
% \end{definition}

% \begin{table}[ht]
%  \centering
%  \caption{論理結合子}
%  \label{tab:logconnect}
%  \begin{tabular}{c l l}
%  \hline
%  記号 & 意味 & \quad 英語 \\ \hline
%  \(\land\) & かつ,連言 & and, conjunction \\
%  \(\lor\)  & または,選言 & or, disjunction \\
%  \(\to\) & ならば,含意 & implies, implication \\
%  \(\neg\) & でない,否定 & not, negation \\
%  \hline
%  \end{tabular}
% \end{table}

% 命題変数の集合$V$を考える.$V$に対する\textbf{真偽値割り当て}$I:V \rightarrow \{1,0\}$と命題論理式$\psi$が与えられたとき,$\psi$の真偽値を以下のように再帰的に定義する.ここで,$I$により$\psi$に真が割り当てられるとき,$I$は$\psi$を\textbf{充足する} (satisfy) といい,$I \models \psi$と表す.また,$I \models \psi$が成り立
% たないことを$I \not\models \psi$と表す.
% \clearpage
% \begin{definition}[充足]\rm
%  以下,$\psi$,$\psi_1$,$\psi_2$を任意の論理式とする.
%  \begin{itemize}
%   \item 命題変数$x$に対して,$I \models x \Leftrightarrow I\left( x \right) = 1$
%   \item $I \models \neg \psi \Leftrightarrow I \not\models \psi$
%   \item $I \models \psi_1 \vee \psi_2 \Leftrightarrow I \models \psi_1$または$I \models \psi_2$
%   \item $I \models \psi_1 \wedge \psi_2 \Leftrightarrow I \models \psi_1$かつ$I \models \psi_2$
%  \end{itemize}
% \end{definition}

\begin{definition}[モデル]\rm
 割り当て$I$が命題論理式$\psi$を充足するとき,$I$は$\psi$の\textbf{モデル(model)}であるという.
\end{definition}
\begin{definition}[充足可能]\rm
 命題論理式$\psi$にモデルが存在するとき,命題論理式$\psi$は\textbf{充足可能(satisfiable)}であるという.
\end{definition}
\begin{definition}[充足不能]\rm
 命題論理式$\psi$にモデルが存在しないとき,命題論理式$\psi$は\textbf{充足不能(unsatisfiable)}であるという.
\end{definition}

\begin{example}\rm
 命題変数の集合を$V=\{ x_1,x_2\}$,命題論理式$\psi$が以下のように与えられているとする.
 \begin{center}
  $\psi ~=~ (x_1 \lor x_2) \land (\neg x_1 \lor x_2)$
 \end{center}
 また,割り当て$I_1,I_2$が以下のように与えられているとする.
 \begin{center}
  $I_1(x_1)=1,I_1(x_2)=0$\\
  $~I_2(x_1)=0,I_2(x_2)=1$
 \end{center}
 このとき,$I_1\models (x_1\lor x_2)$は真であるが,$I_1\models (\neg x_1\lor x_2)$は偽であるので,$I_1\not\models \psi$.\\
 一方で,$I_2\models (x_1\lor x_2)$,$I_2\models (\neg x_1\lor x_2)$は共に真であるので,$I_2\models \psi$より,$I_2$は$\psi$のモデルである. すなわち,$\psi$は充足可能である.
\end{example}

\section{CNFの定義}
\label{sec:defCNF}
SAT問題においては,以下で定義する\textbf{連言標準形(Conjunctive Normal Form;CNF)}を入力とすることが一般的であり,本論文でもSAT問題はCNFで与えられるものとする.\\
\begin{definition}[リテラル]\rm
 \textbf{リテラル(literal)}とは,命題変数 $x$ またはその否定 $\neg x$ であり,前者を正リテラル,後者を負リテラルと呼ぶ.
\end{definition}
\begin{definition}[節]\rm
 \textbf{節(clause)}とは,リテラルの選言 ($\vee$;OR) ,すなわち,リテラルを論理和$\vee$で結合した論理式である.
\end{definition}
\begin{definition}[CNF]\rm
 \textbf{CNF(Conjunctive Normal Form;連言標準形)}とは,節の連言($\wedge$;AND),すなわち,節を論理積$\wedge$で結合した論理式である.
\end{definition}

\begin{example}\rm
 \label{ex:CNF}
 CNF$\psi$が以下のように与えられているとする:\\
 \begin{center}
  $(x_1\lor x_2)\land (x_1\lor \neg x_2)\land (x_1 \lor \neg x_3) \land (x_2\lor x_3)$
 \end{center}
 このとき,割り当て$(x_1,x_2,x_3)=(1,0,1)$は$\psi$を充足するため,$\psi$は充足可能である(SAT).他の$\psi$のモデルとして,$(1,1,0)$と$(1,1,1)$が存在する.\\
 一方,CNF$\psi' =\psi \land (\neg x_1)$は充足不能となる(UNSAT).なぜなら,$\psi$の任意のモデル$I$において,$I(x_1)=1$であり,$I\not\models \neg x_1$であるため,$I \not\models \psi'$となるからである.
\end{example}
\section{極大モデル,極小モデルの定義}
\label{sec:defmaxmin}
本節では,論文\cite{長谷川隆三2010モデル列挙とモデル計数}をもとに極大モデル,極小モデルを定義する.\\
CNFのモデルは,そのモデルで真と解釈される変数の集合で表現することができる.例えば,$x_1$を真, $x_2$を偽,$x_3$を真,と解釈するモデルは変数集合$\{x_1,x_3\}$で表現できる.このように表すことにより,特定のモデル間に包含関係が導入され,モデル間の大小関係を自然に導入できる.例えば,モデル$\{ x_1,x_3\}$はモデル$\{ x_1,x_2,x_3\}$より包含関係上で小さい.繰り返しになるが,モデル間にこのような包含関係を導入したとき,極大モデル,極小モデルを以下のようにも表すことができる.
また,本論文では,理解のしやすさのためモデルMを次のように定義する.
\begin{definition}[モデルM]\rm
 命題論理式$\psi$のモデル$I:V\to \{ 0,1\}$に対し,モデルMを以下のように定義する.
 \begin{center}
  $M~=~\{ v \mid I(v)=1\}$
 \end{center}
\end{definition}
\begin{definition}\rm
 $M_1$, $M_2$を変数集合とする.このとき,$M_1$が$M_2$より大きいとは, $M_2$が $M_1$の真部分集合であることをいう.
\end{definition}
\begin{definition}[極大モデル]\rm
 $\psi$を命題論理式, $M$を$\psi$のモデルとする. このとき, $M$が$\psi$の\textbf{極大モデル}である, とは,$M$より大きい$\psi$のモデルが存在しないことをいう.
\end{definition}
\begin{definition}\rm
$M_1$, $M_2$を変数集合とする.このとき,$M_1$が$M_2$より小さいとは, $M_1$が $M_2$の真部分集合であることをいう.
\end{definition}
\begin{definition}[極小モデル]\rm
$\psi$を命題論理式, $M$を$\psi$のモデルとする. このとき, $M$が$\psi$の\textbf{極小モデル}である, とは,$M$より小さい$\psi$のモデルが存在しないことをいう.
\end{definition}
\begin{example}\rm
 \ref{sec:defCNF}節の例\ref{ex:CNF}で挙げたCNF$\psi$の場合,極小モデル,極大モデルは表\ref{tab:psi-models}のようになる.
 \begin{table}[ht]
  \centering
   \caption{$\psi$のモデルと極小・極大モデル}
   \label{tab:psi-models}
   \begin{tabular}{c c c}
    \toprule
    モデル$(x_1,x_2,x_3)$ & 真になる変数集合 & 区分 \\ 
    \midrule
     $(1,0,1)$ & $\{x_1,x_3\}$ & 極小モデル \\
     $(1,1,0)$ & $\{x_1,x_2\}$ & 極小モデル \\
     $(1,1,1)$ & $\{x_1,x_2,x_3\}$ & 極大モデル \\
    \bottomrule
   \end{tabular}
 \end{table}
 今回の例のモデルは極小,極大モデルのみであったが,一般的な問題では極小,極大モデルのどちらでもないモデルが存在する.
\end{example}

 % \ref{sec:defCNF}節の例\ref{ex:CNF}で挙げたCNF$\psi$の場合,$\psi$のモデルは$(x_1,x_2,x_3)=(1,0,1),(1,1,0),(1,1,1)$であった.このとき,$(x_1,x_2,x_3)=(1,0,1),(1,1,0)$は$(x_1,x_2,x_3)=(1,1,1)$の真部分集合であるため,$(x_1,x_2,x_3)=(1,1,1)$は$(x_1,x_2,x_3)=(1,0,1),(1,1,0)$より大きい.また,$(x_1,x_2,x_3)=(1,1,1)$より大きい$\psi$のモデルは存在しないので,$(x_1,x_2,x_3)=(1,1,1)$は$\psi$の極大モデルである.\\
 % また,$(x_1,x_2,x_3)=(1,0,1),(1,1,0)$より小さい$\psi$のモデルは存在しないので,$(x_1,x_2,x_3)=(1,0,1),(1,1,0)$は$\psi$の極小モデルである.

  %%%%%%%%%%%%%%%%%%%%%%%%%%%%%%%%%%%%%%%%%%%%%%%%%%%%%%%%%% 








%%%%%%%%%%%%%%%%%%%%%%%%%%%%%%%%%%%%%%%%%%%%%%%%%%%%%%%%%% 
\section{SATソルバー複数回起動による極小・極大モデル計算の既存研究} \label{chap:koshimura}
%%%%%%%%%%%%%%%%%%%%%%%%%%%%%%%%%%%%%%%%%%%%%%%%%%%%%%%%%% 

本章では,論文\cite{koshimura2009minimal}で提案された,SATソルバーを複数回起動することでSAT問題の極小モデルを求める方法について説明する.
ここでSAT問題を符号化した$P$が与えられたとき,極小モデルは以下の手順で求めることができる.
\begin{enumerate}
 \item $P$にSATソルバーを起動し,モデルMを求める
 \item 1.で求めたモデルMのうち,真として含まれるリテラルを$x_1,...,x_m$とし,偽として含まれるリテラルを$y_1,...,y_n$とし,次のF1,F2を作る
  \begin{center}
  F1$~=~\neg (x_1 \land ... \land x_m)$\\
  F2$~=~\neg y_1 \land ... \land \neg y_n$
  \end{center}
 \item $P \coloneq  P \land F1 \land F2$として,SATソルバーを起動する
 \begin{itemize}
  \item UNSATならば,モデルMが極小モデル
  \item SATならば,1.に戻る
 \end{itemize}
\end{enumerate}
アルゴリズムで記述するとAlgorithm\ref{alg:minimal-model}のようになる.

\begin{algorithm}[t]
 \caption{SATソルバー複数回起動による極小モデル生成}
 \label{alg:minimal-model}
 \begin{algorithmic}[1]
  \Require SAT問題の符号化 $P$
  \Ensure $P$ の極小モデル $M$(存在しなければ UNSAT)

  \While{\textsc{solve}$(P)$ が SAT を返し,モデル $M$ を得る}
    \State $X \gets \{x \mid x \text{ は } M \text{ において真となるリテラル}\}$
    \State $Y \gets \{y \mid y \text{ は } M \text{ において偽となるリテラル}\}$
    \State \Comment{$X=\{x_1,\dots,x_m\},\;Y=\{y_1,\dots,y_n\}$ とする}
    \State $F1 \gets \neg(x_1 \land \cdots \land x_m)$ \Comment{$\equiv (\neg x_1 \lor \cdots \lor \neg x_m)$}
    \State $F2 \gets (\neg y_1) \land \cdots \land (\neg y_n)$
    \State $P' \gets P \land F1 \land F2$
    \If{\textsc{solve}$(P')$ が UNSAT}
        \State \textbf{return} $M$
    \Else
        \State $P \gets P'$
    \EndIf
  \EndWhile
  \State \textbf{return} UNSAT
 \end{algorithmic}
\end{algorithm}

\begin{example}\rm
 \ref{sec:defCNF}節の例\ref{ex:CNF}で与えられた式を$P$として,先ほどのアルゴリズムに沿って極小モデルを求めてみる.\\
(1回目のループ)
\begin{enumerate}
 \item $P$にSATソルバーを起動し,$(x_1,x_2,x_3)=(1,1,1)$がモデルMとして求められる
 \item 1.で求めたモデルMから次のF1,F2を作る
  \begin{center}
  F1$~=~\neg (x_1 \land x_2 \land x_3)$\\
  F2$~=~\top$
  \end{center}
 \item $P \coloneq P ~\land ~\neg (x_1 \land x_2 \land x_3)~ \land ~\top$として,SATソルバーを起動すると,SATであるので,1.に戻る.
\end{enumerate}
(2回目のループ)
\begin{enumerate}
 \item $P \coloneq P ~\land ~\neg (x_1 \land x_2 \land x_3)~ \land ~\top$にSATソルバーを起動すると,$(x_1,x_2,x_3)=(1,1,0)$がモデルMとして求められる.
 \item 1.で求めたモデルMから次のF1,F2を作る
  \begin{center}
  F1$~=~\neg (x_1 \land x_2)$\\
  F2$~=~\neg x_3$
  \end{center}
 \item $P \coloneq P \land \neg (x_1 \land x_2) \land \neg x_3$として,SATソルバーを起動すると,UNSATであるので$(x_1,x_2,x_3)=(1,1,0)$は極小モデルである.
\end{enumerate}
以上のようにして,極小モデルが求まる.\\
ただし,上の例の1.で求まるモデルMはいくつかあり,あくまで一例である.上のような操作を繰り返し行うことで,全ての極小モデルを列挙することができる.\\
また,先ほどの手順2.におけるF1,F2を以下のように変更すれば,極大モデルを求めることができる.
\begin{center}
  F1$~=~\neg (\neg y_1 \land ... \land \neg y_n)$\\
  F2$~=~x_1 \land ... \land x_m$
\end{center}
極小モデルでは,いま見つけたモデルMより真となるリテラルが少ないモデルが存在するかをSATソルバーで調べ、存在しなければモデルMを極小モデルとして返すという計算方法であったので,極大モデルでは,いま見つけたモデルMより偽となるリテラルが少ないモデルが存在するかをSATソルバーで調べ、存在しなければモデルMを極大モデルとして返すという計算方法にしている.\\

% $(x_1,x_2,x_3)=(1,1,1)$がモデルMとして求められたとき,$F1=\neg (x_1 \land x_2 \land x_3)$,$F2=\top$となり,$P \coloneq P ~\land ~\neg (x_1 \land x_2 \land x_3)~ \land ~\top$となる.これはSATであるので,1.に戻り,新たに定義した$P$についてSATソルバーを起動すると,$(x_1,x_2,x_3)=(1,1,0)$または,$(x_1,x_2,x_3)=(1,0,1)$を返す.$(x_1,x_2,x_3)=(1,1,0)$が返ってきたとし,このモデルをMとすると,$P$は新たに$P \coloneq P \land \neg (x_1 \land x_2) \land \neg x_3$となる.これはUNSATであるので極小モデルは$(x_1,x_2,x_3)=(1,1,0)$と求まる.

\end{example}


%%%%%%%%%%%%%%%%%%%%%%%%%%%%%%%%%%%%%%%%%%%%%%%%%%%%%%%%%% 
\chapter{SAT符号化を用いたSATソルバー1回起動による極小・極大モデル計算方法} \label{chap:maxcalc}
%%%%%%%%%%%%%%%%%%%%%%%%%%%%%%%%%%%%%%%%%%%%%%%%%%%%%%%%%% 

\ref{chap:koshimura}章では,SATソルバーを複数回起動することによって極小・極大モデルを計算する方法を紹介した.ただ,SATソルバーを複数回起動するのは手間がかかってしまう.~論文\cite{adachi2023}では,SAT問題に符号化を施したCNFに変換を施して,SATソルバーを1回起動することで,極小モデルを計算する方法が提案されたが,計算方法の実装や評価は示されていなかった.そこで,本論文では,SATソルバー1回の起動で極大モデルを計算できるようなCNFへの変換方法を提案し,極小モデルの計算方法とともに実装,評価する.また,本章ではSAT問題に符号化を施したCNFに対して変換を施し,SATソルバーを1回起動することで,極小・極大モデルを計算できる変換方法について説明する.
\section{SATソルバー1回起動による極小モデル計算の先行研究}
\label{sec:mintrans}
\paragraph{極小モデル計算のためのCNF変換方法}
\label{para:mintrans}

本節では,論文\cite{adachi2023}で提案された,SAT問題に符号化を施したCNFの正リテラルに注目し,できる限り真となる変数を減らしたいという動機のもとCNFに新たな制約を追加し,SATソルバー1回起動によって極小モデルを計算する手法について説明する.この計算手法では,ある節中の他の変数で節が満たされている時,その変数を偽にするという制約を追加し,極小モデルを求める.
\paragraph{CNFの変換式}
\label{para:minform}

与えられたCNFを$\Phi $,変換後のCNFを$\Omega _{min} $ とすると,提案変換は,$\Phi$に以下のような制約$M_{min}(\Phi )$ を追加し,$\Omega _{min} = \Phi \land M_{min}(\Phi )$ とするものである.
\begin{formula}
 \label{fom:min}
\begin{align*}
  &M_{min}(\Phi) \equiv \bigwedge_{x \in Var(\Phi)} Cl_{min}(\Phi,x) \rightarrow \neg x\\
  &Cl_{min}(\Phi,x) \equiv \bigwedge_{c \in \Phi,x \in c}c \backslash \{x\}\\
\end{align*}
\end{formula}
ここで,$Cl_{min}(\Phi,x)$は,変数$x$が正リテラルとして含まれる全ての節から$x$を除いたものの連言であり,$M_{min}(\Phi )$は全ての変数$x$において,$Cl_{min}(\Phi,x)$が満たされるならば,$x$を偽とするという制約の連言である.
\begin{example}\rm
 以下のようなCNF式$\Phi$が与えられているとする.
 \begin{center}
  $\Phi ~=~ (\neg x_1 \lor \neg x_2 \lor \neg x_3) \land (x_1 \lor \neg x_2) \land (x_1 \lor x_2 \lor \neg x_3)$
 \end{center}
 このとき,$Cl_{min}(\Phi,x_1)$は,各節から正リテラルとして含まれる変数$x_1$を除いたものの連言であるので,以下のようになる.
 \begin{center}
  $Cl_{min}(\Phi , x_1)~=~\neg x_2 \land (x_2 \lor \neg x_3)$
 \end{center}
 すると,$M_{min}(\Phi )$により,$Cl_{min}(\Phi,x_1)$が満たされるならば,$x_1$を偽とするという以下の制約が追加される.
 \begin{center}
  $(\neg x_2 \land (x_2 \lor \neg x_3))\to \neg x_1$
 \end{center}
この制約は,$(x_2,x_3)~=~(0,0)$の時,$x_1$は偽になるということを意味する.
\end{example}


\section{SATソルバー1回起動による極大モデル計算方法}
\label{sec:maxtrans}
\paragraph{極大モデル計算のためのCNF変換方法}
\label{para:maxtrans}

\ref{sec:mintrans}節では,SAT問題に符号化を施したCNFの正リテラルに注目し,できる限り真となる変数を減らしたいという動機のもとCNFに新たな制約を追加し,極小モデルを計算した.本節では,SAT問題に符号化を施したCNFの負リテラルに注目し,できる限り偽となる変数を減らしたいという動機のもとCNFに新たな制約を追加し,極大モデルを計算する方法を提案する.ある節中の他の変数で節が満たされている時,その変数を真にするというのが変換によって追加される制約の意図である.

\paragraph{CNFの変換式}
\label{para:maxform}

与えられたCNFを$\Phi$,変換後のCNFを$\Omega _{max} $ とすると,提案変換は,$\Phi$に以下のような制約$M_{max}(\Phi )$ を追加し,$\Omega _{max} = \Phi \land M_{max}(\Phi )$ とするものである.
\begin{formula}
 \label{fom:max}
\begin{align*}
  &M_{max}(\Phi) \equiv \bigwedge_{x \in Var(\Phi)} Cl_{max}(\Phi,\neg x) \rightarrow x\\
  &Cl_{max}(\Phi,\neg x) \equiv \bigwedge_{c \in \Phi,\neg x \in c}c \backslash \{\neg x\}\\
\end{align*}
\end{formula}
ここで,$Cl_{max}(\Phi,\neg x)$は,変数$x$が負リテラルとして含まれる全ての節から$\neg x$を除いたものの連言であり,$M_{max}(\Phi )$は全ての変数$x$において,$Cl_{max}(\Phi,\neg x)$が満たされるならば,$x$を真とするという制約の連言である.
\begin{example}\rm
 以下のようなCNF式$\Phi$が与えられているとする.
 \begin{center}
  $\Phi ~=~ (x_1 \lor x_2 \lor x_3) \land (\neg x_1 \lor x_2) \land (\neg x_1 \lor \neg x_2 \lor x_3)$
 \end{center}
 このとき,$Cl_{max}(\Phi,\neg x_1)$は,各節から負リテラルとして含まれる変数$x_1$を除いたものの連言であるので,以下のようになる.
 \begin{center}
  $Cl_{max}(\Phi , \neg x_1)~=~ x_2 \land (\neg x_2 \lor x_3)$
 \end{center}
 すると,$M_{max}(\Phi )$により,$Cl_{max}(\Phi,\neg x_1)$が満たされるならば,$x_1$を真とするという以下の制約が追加される.
 \begin{center}
  $(x_2 \land (\neg x_2 \lor x_3))\to x_1$
 \end{center}
この制約は,$(x_2,x_3)~=~(1,1)$の時,$x_1$は真になるということを意味する.
\end{example}

\paragraph{極小モデル,極大モデル計算の例と計算過程}
\label{para:mincalc}

\begin{example}[極小モデル計算の例]\rm
\label{ex:minPhi}
 以下のようなCNF式$\Phi$が与えられているとする.
 \begin{center}
  $\Phi ~=~ (x_1 \lor \neg x_2 \lor x_3) \land (x_2 \lor x_3) \land (x_3 \lor \neg x_4)$
 \end{center}
 このCNF$\Phi$のモデルは以下のようになる.
 \begin{center}
  $(x_1,x_2,x_3,x_4)=(1,0,1,0)$,$(1,1,1,0)$,$(0,0,1,0)$,$(0,1,1,0)$,$(1,0,1,1)$,$(0,0,1,1)$,$(1,1,1,1)$,$(0,1,1,1)$,$(1,1,0,0)$
 \end{center}
 このCNF$\Phi$に式\ref{fom:min}に沿って変換を施すと,表\ref{tab:Phimin}のような制約が追加され,最終的に$(x_1,x_2,x_3,x_4)~=~(0,0,1,0),(1,1,0,0)$という極小モデルが計算でき,これは$\Phi$の極小モデルに一致する.\\

\renewcommand{\arraystretch}{1.2}
\begin{table}[H]
  \centering
  \small
  \setlength{\tabcolsep}{3pt}
  \caption{CNF$\Phi$の極小変換}
  \label{tab:Phimin}

  \begin{tabularx}{\linewidth}{c | >{\raggedright\arraybackslash}X
                                  >{\raggedright\arraybackslash}X
                                  >{\raggedright\arraybackslash}X}
    \toprule
    注目変数 & 追加する節(制約) & 制限されるモデル & 極小モデル候補\\
    \midrule

    $x_1$
    & $(\neg x_2 \lor x_3)\to \neg x_1$
    & $\begin{aligned}
        &(1,0,1,0),(1,1,1,0),\\
        &(1,0,1,1),(1,1,1,1)
       \end{aligned}$
    & $\begin{aligned}
        &(0,0,1,0),(0,1,1,0),\\
        &(0,0,1,1),(0,1,1,1),\\
        &(1,1,0,0)
       \end{aligned}$
    \\
    \hline

    $x_2$
    & $x_3 \to \neg x_2$
    & $\begin{aligned}
        &(1,1,1,0),(0,1,1,0),\\
        &(1,1,1,1),(0,1,1,1)
       \end{aligned}$
    & $\begin{aligned}
        &(0,0,1,0),(0,0,1,1),\\
        &(1,1,0,0)
       \end{aligned}$
    \\
    \hline

    $x_3$
    & $\begin{aligned}
        &{\scriptstyle((x_1 \lor \neg x_2)\land x_2 \land \neg x_4)}\\
        &\to \neg x_3
       \end{aligned}$
    & $(1,1,1,0)$
    & $\begin{aligned}
        &(0,0,1,0),(0,0,1,1),\\
        &(1,1,0,0)
       \end{aligned}$
    \\
    \hline

    $x_4$
    & $\top \to \neg x_4\ (\equiv \neg x_4)$
    & $\begin{aligned}
        &(1,0,1,1),(0,0,1,1),\\
        &(1,1,1,1),(0,1,1,1)
       \end{aligned}$
    & $\begin{aligned}
        &(0,0,1,0),(1,1,0,0)
       \end{aligned}$
    \\
    \bottomrule
  \end{tabularx}
\end{table}

 % \renewcommand{\arraystretch}{1.4}
 %  \begin{table}[H]
 %  \centering
 %   \caption{CNF$\Phi$の極小変換}
 %   \label{tab:Phimin}
 %   \begin{tabular}{c | >{\raggedright\arraybackslash}p{4cm}
 %                   >{\raggedright\arraybackslash}p{4cm}
 %                   >{\raggedright\arraybackslash}p{5.3cm}}
 %    \toprule
 %    注目変数 &  追加する節(制約) & 制限されるモデル & 極小モデル候補\\ 
 %    \midrule
 %     $x_1$ & $(\neg x_2 \lor x_3) \to \neg x_1$ 
 %       & $
 %       \begin{aligned}
 %        &(1,0,1,0),(1,1,1,0),\\
 %        &(1,0,1,1),(1,1,1,1) 
 %       \end{aligned}$
 %       & $
 %        \begin{aligned}
 % 	&(0,0,1,0),(0,1,1,0),\\
 %        &(0,0,1,1),(0,1,1,1),(1,1,0,0)
 % 	\end{aligned}$ \\
 %     \hline
 %     $x_2$ & $x_3 \to \neg x_2$ 
 %       & $
 %        \begin{aligned}
 % 	 &(1,1,1,0),(0,1,1,0),\\
 %         &(1,1,1,1),(0,1,1,1)
 % 	\end{aligned}$
 %       & $
 %       \begin{aligned}
 % 	&(0,0,1,0),(0,0,1,1),\\
 %        &(1,1,0,0)
 %       \end{aligned}$ \\
 %     \hline
 %     $x_3$ & $((x_1 \lor \neg x_2)\land x_2 \land \neg x_4) \to \neg x_3$ 
 %       & $(1,1,1,0)$ 
 %       & $
 %        \begin{aligned}
 % 	 &(0,0,1,0),(0,0,1,1),\\
 %         &(1,1,0,0)
 % 	\end{aligned}$ \\
 %     \hline
 %     $x_4$ & $\top \to \neg x_4~~(\equiv \neg x_4)$ 
 %       & $
 %        \begin{aligned}
 % 	 &(1,0,1,1),(0,0,1,1),\\
 %         &(1,1,1,1),(0,1,1,1), 
 % 	\end{aligned}$ 
 %       & $(0,0,1,0),(1,1,0,0)$\\
 %    \bottomrule
 %   \end{tabular}
 % \end{table}
表\ref{tab:Phimin}で行われた極小モデル計算をさらに詳しく解説する.まず変数$x_1$に注目すると,1つ目の節で正リテラルとして出現しているため,変数$x_1$以外の変数で他の節が真である時,変数$x_1$が偽となる制約$(\neg x_2 \wedge x_3) \rightarrow \neg x_1$を追加する.次に変数$x_2$に注目すると,2つ目の節で正リテラルとして出現しているため,変数$x_2$以外の変数で他の節が真である時,変数$x_2$が偽となる制約$x_3 \rightarrow \neg x_2$を追加する.次に変数$x_3$に注目すると,すべての節で正リテラルとして出現している.このとき,全ての節において,変数$x_3$以外の変数で節が真である時,変数$x_3$が偽となる制約$((x_1 \lor \neg x_2)\land x_2 \land \neg x_4) \to \neg x_3$を追加する.次に変数$x_4$に注目すると,変数$x_4$を正リテラルとして含む節は存在しない.すなわち,$x_4$が偽であっても全ての節が充足されないことはないため,単位節$\neg x_4$を追加する.\\
このようにして,CNFに変換を加え,新たな制約を加えることで極小モデルを計算することができる.
\end{example}

\begin{example}[極大モデル計算の例]\rm
\label{ex:Phimax}
 以下のようなCNF式$\Phi$が与えられているとする.
 \begin{center}
  $\Phi ~=~ (\neg x_1 \lor x_2 \lor \neg x_3) \land (\neg x_2 \lor \neg x_3) \land (\neg x_3 \lor x_4)$
 \end{center}
 このCNF$\Phi$のモデルは以下のようになる.
 \begin{center}
  $(x_1,x_2,x_3,x_4)=(0,0,0,0)$,$(0,0,0,1)$,$(0,0,1,1)$,$(0,1,0,0)$,\\
$(0,1,0,1)$,$(1,0,0,0)$,$(1,0,0,1)$,$(1,1,0,0)$,$(1,1,0,1)$
 \end{center}
 このCNF$\Phi$に式\ref{fom:max}に沿って変換を施すと,表\ref{tab:Phimax}のような制約が追加され,最終的に$(x_1,x_2,x_3,x_4)~=~(0,0,1,1),(1,1,0,1)$という極大モデルが計算でき,これらは$\Phi$の極大モデルに一致する.\\

\renewcommand{\arraystretch}{1.2}
\begin{table}[H]
  \centering
  \small % ここで全体を少し小さく
  \setlength{\tabcolsep}{3pt} % 列間を詰める(必要なら2ptまで)
  \caption{CNF$\Phi$の極大変換}
  \label{tab:Phimax}

  % 全体幅を \linewidth に固定
  \begin{tabularx}{\linewidth}{c | >{\raggedright\arraybackslash}X
                                  >{\raggedright\arraybackslash}X
                                  >{\raggedright\arraybackslash}X}
    \toprule
    注目変数 & 追加する節(制約) & 制限されるモデル & 極大モデル候補 \\
    \midrule

    $x_1$
    & $(x_2 \lor \neg x_3)\to x_1$
    & $\begin{aligned}
        &(0,0,0,0),(0,0,0,1),\\
        &(0,1,0,0),(0,1,0,1)
       \end{aligned}$
    & $\begin{aligned}
        &(0,0,1,1),(1,0,0,0),\\
        &(1,0,0,1),(1,1,0,0),\\
        &(1,1,0,1)
       \end{aligned}$
    \\
    \hline

    $x_2$
    & $\neg x_3 \to x_2$
    & $\begin{aligned}
        &(0,0,0,0),(0,0,0,1),\\
        &(1,0,0,0),(1,0,0,1)
       \end{aligned}$
    & $\begin{aligned}
        &(0,0,1,1),(1,1,0,0),\\
        &(1,1,0,1)
       \end{aligned}$
    \\
    \hline

    $x_3$
    % 長い式はセル内で明示的に改行
    & $\begin{aligned}
        &{\scriptstyle ((\neg x_1 \lor x_2)\land \neg x_2 \land x_4)}\\
        &\to x_3
       \end{aligned}$
    & $(0,0,0,1)$
    & $\begin{aligned}
        &(0,0,1,1),(1,1,0,0),\\
        &(1,1,0,1)
       \end{aligned}$
    \\
    \hline

    $x_4$
    & $\top \to x_4\ (\equiv x_4)$
    & $\begin{aligned}
        &(0,0,0,0),(0,1,0,0),\\
        &(1,0,0,0),(1,1,0,0)
       \end{aligned}$
    & $(0,0,1,1),(1,1,0,1)$
    \\
    \bottomrule
  \end{tabularx}
\end{table}

 % \renewcommand{\arraystretch}{1.4}
 %  \begin{table}[H]
 %  \centering
 %   \caption{CNF$\Phi$の極大変換}
 %   \label{tab:Phimax}
 %   \begin{tabular}{c | >{\raggedright\arraybackslash}p{4cm}
 %                   >{\raggedright\arraybackslash}p{4cm}
 %                   >{\raggedright\arraybackslash}p{5.3cm}}
 %    \toprule
 %    注目変数 &  追加する節(制約) & 制限されるモデル & 極大モデル候補\\ 
 %    \midrule
 %     $x_1$ & $(x_2 \lor \neg x_3) \to x_1$ 
 %       & $
 %       \begin{aligned}
 %        &(0,0,0,0),(0,0,0,1),\\
 %        &(0,1,0,0),(0,1,0,1) 
 %       \end{aligned} $
 %       &$(1,0,0,0),(1,0,0,1),(1,1,0,1)$\\
 %     \hline
 %     $x_2$ & $\neg x_3 \to x_2$ 
 %       & $
 %        \begin{aligned}
 % 	 &(0,0,0,0),(0,0,0,1),\\
 %         &(1,0,0,0),(1,0,0,1)
 % 	\end{aligned} $
 %       &(1,1,0,1)\\
 %     \hline
 %     $x_3$ & $((\neg x_1 \lor x_2)\land \neg x_2 \land x_4) \to x_3$ 
 %       & $(0,0,0,1)$ 
 %       & $(1,1,0,1)$\\
 %     \hline
 %     $x_4$ & $\top \to x_4~~(\equiv x_4)$ 
 %       &$(0,0,0,0),(0,1,0,0)$
 %       &$(1,1,0,1)$\\
 %    \bottomrule
 %   \end{tabular}
 % \end{table}
表\ref{tab:Phimax}で行われた極大モデル計算をさらに詳しく解説する.まず変数$x_1$に注目すると,1つ目の節で負リテラルとして出現しているため,変数$x_1$以外の変数で他の節が真である時,変数$x_1$が真となる制約$(x_2 \wedge \neg x_3) \rightarrow x_1$を追加する.次に変数$x_2$に注目すると,2つ目の節で負リテラルとして出現しているため,変数$x_2$以外の変数で他の節が真である時,変数$x_2$が真となる制約$\neg x_3 \rightarrow x_2$を追加する.次に変数$x_3$に注目すると,すべての節で負リテラルとして出現している.このとき,全ての節において,変数$x_3$以外の変数で節が真である時,変数$x_3$が真となる制約$((\neg x_1 \lor x_2)\land \neg x_2 \land x_4) \to x_3$を追加する.次に変数$x_4$に注目すると,変数$x_4$を負リテラルとして含む節は存在しない.すなわち,$x_4$が真であっても全ての節が充足されないことはないため,単位節$x_4$を追加する.\\
このようにして,CNFに変換を加え,新たな制約を加えることで極大モデルを計算することができる.
\end{example}
%%%%%%%%%%%%%%%%%%%%%%%%%%%%%%%%%%%%%%%%%%%%%%%%%%%%%%%%%% 
\section{提案手法の実装} \label{chap:maxminalgo}
%%%%%%%%%%%%%%%%%%%%%%%%%%%%%%%%%%%%%%%%%%%%%%%%%%%%%%%%%% 

\section{変換アルゴリズム}
\label{sec:algo}
極小変換,極大変換の一般的なアルゴリズムをAlgorithm\ref{alg:trans-minmax}に記述する.アルゴリズムでは,入力をCNF$\Phi$とmin,maxとし,入力によって極小変換と極大変換に分岐できる.また出力は変換後のCNF$\Omega$である.

CNF $\Phi$ から変換後の CNF $\Omega$ を構成する手順を,行番号に沿って説明する.Algorithm \ref{alg:trans-minmax} は入力として CNF $\Phi$ と mode $\in\{\mathrm{min},\mathrm{max}\}$ を受け取り,補助制約 $M$ を生成したうえで $\Omega \gets \Phi \land M$ を返す(行1--2, 28--29).

\paragraph{準備(行1--2)}
まず $\Phi$ に出現する命題変数全体を $P \gets \mathrm{Var}(\Phi)$ として取り出す(行1).続いて,追加制約を保管するための CNF を $M \gets \top$ と初期化する(行2).

\paragraph{mode による分岐(行3--11)}
各変数 $p \in P$ について,その変数に関する「極小/極大らしさ」を強制する制約を追加する(行3).このとき,まず $c_1 \gets \bot$ を用意し(行4),mode に応じて 2 つのリテラル $t$ と $h$ を次のように選ぶ(行5--11).
\begin{itemize}
  \item mode = min のとき: $t \gets p,\ \ h \gets \neg p$(行6--7).
  \item mode = max のとき: $t \gets \neg p,\ \ h \gets p$(行9--10).
\end{itemize}
ここで $t$ は「節の中で注目するリテラル」,$h$ は最終的に追加する含意 $(h \lor c_1)$ の左側に出るリテラル(行26)である.この入れ替えにより,極小変換と極大変換を同一の枠組みで処理できる.

\paragraph{各節に対する補助変数の導入(行12--25)}
次に $\Phi$ の各節 $clause \in \Phi$ を走査し(行12),その節が $t$ を含む場合のみ処理を行う(行13).\\
$t \in clause$ であるとき,新しい補助変数 $x$ を 1 つ導入し(行14),さらに $c_2 \gets \bot$ を初期化する(行15).\\
その後,節 $clause$ 内の各リテラル $l \in clause$ を走査し(行16),$l \neq t$ のものだけを用いて以下を行う(行17--20).\\
\begin{enumerate}
  \item 制約 $(\neg x \lor \neg l)$ を $M$ に追加する(行18).
  \item $c_2 \gets c_2 \lor l$ として,$t$ 以外のリテラルの選言 $c_2$ を作る(行19).
\end{enumerate}
このとき,$(\neg x \lor \neg l)$ は $x \to \neg l$ を表すため,$x$ が真なら「$t$ 以外のリテラルはすべて偽である」ことを要求する.

ループ後(行21),さらに $(c_2 \lor x)$ を $M$ に追加する(行22).
これは $(\neg c_2) \to x$ を表し,節中の他リテラル($t$ 以外)がすべて偽(すなわち $c_2$ が偽)なら $x$ を真にせよ,という意味になる.結果として,$x$ は
\[
x \ \leftrightarrow\  \bigwedge_{l \in clause,\ l\neq t} \neg l
\]
(「その節で $t$ 以外がすべて偽である」こと)を CNF で表す指示変数として機能する.最後に $c_1 \gets c_1 \lor x$ として(行23),$t$ を含む各節について生成した $x$ の選言を $c_1$ に保管する(行23).

\paragraph{変数ごとの主要制約の追加(行26)}
全節の走査が終わると,$M \gets M \land (h \lor c_1)$ を追加する(行26).\\
$c_1$ は「$t$ を含むどれかの節で,$t$ 以外がすべて偽になる(=その節が $t$ に依存する)」ことを表す.したがって $(h \lor c_1)$ は
\[
\neg h \to c_1
\]
を意味し,\textbf{$h$ を偽にする(min なら $p$ を真にする,max なら $p$ を偽にする)場合には,その選択が必須である状況($c_1$ が真)を要求する}
という形で,極小性,極大性を CNF 制約として埋め込む.

\paragraph{出力(行28--29)}
以上で得た $M$ を元の CNF に連言し,$\Omega \gets \Phi \land M$ を返す(行28--29).\\
この $\Omega$ を SAT ソルバーで 1 回解くことで,mode に応じた極小モデル,極大モデルに対応する解を得る(Algorithm 2 の目的).


\begin{algorithm}[t]
\caption{CNF $\Phi$ からCNF への変換方法$\Omega$(min/max)}
\label{alg:trans-minmax}
\begin{algorithmic}[1]
\Require CNF $\Phi$, mode $\in \{\mathrm{min},\mathrm{max}\}$
\Ensure  CNF $\Omega$
\State $P \gets \mathrm{Var}(\Phi)$
\State $M \gets \top$ %\Comment{additional constraints (empty conjunction)}
\ForAll{$p \in P$}
  \State $c_1 \gets \bot$ %\Comment{a disjunction (empty disjunction)}
  \If{$mode = \mathrm{min}$}
    \State $t \gets p$ %\Comment{focus on positive literal $p$ (Eq.1)}
    \State $h \gets \neg p$ %\Comment{final clause is $(h \vee c_1) = (\neg p \vee c_1)$}
  \Else
    \State $t \gets \neg p$ %\Comment{focus on negative literal $\neg p$ (Eq.2)}
    \State $h \gets p$ %\Comment{final clause is $(h \vee c_1) = (p \vee c_1)$}
  \EndIf

  \ForAll{$clause \in \Phi$}
    \If{$t \in clause$}
      \State $x \gets$ new variable
      \State $c_2 \gets \bot$ %\Comment{$c_2 = \bigvee (clause \setminus \{t\})$}
      \ForAll{$l \in clause$}
        \If{$l \neq t$}
          \State $M \gets M \wedge (\neg x \vee \neg l)$ %\Comment{$x \rightarrow \neg l$}
          \State $c_2 \gets c_2 \vee l$
        \EndIf
      \EndFor
      \State $M \gets M \wedge (c_2 \vee x)$ %\Comment{$\neg c_2 \rightarrow x$}
      \State $c_1 \gets c_1 \vee x$
    \EndIf
  \EndFor

  \State $M \gets M \wedge (h \vee c_1)$
\EndFor
\State $\Omega \gets \Phi \wedge M$
\State \textbf{return} $\Omega$
\end{algorithmic}
\end{algorithm}

%=========================
\section{変換アルゴリズムの動作例}
\label{subsec:example-trans-minmax}

本節では,Algorithm~\ref{alg:trans-minmax} が具体的にどのように新規変数を追加し,CNFを変換するかを,小さなサイズの CNF を入力として示す.

\paragraph{入力 CNF}
次の CNF を入力 $\Phi$ とする:
\begin{align}
\Phi \;=\;& (p \lor q)\ \land\ (p \lor \neg q)\ \land\ (\neg p \lor r).
\label{eq:phi-example}
\end{align}
出現変数は $P=\mathrm{Var}(\Phi)=\{p,q,r\}$ である.

%------------------------------------------------------------
\paragraph{極小変換(mode=min)}
Algorithm~\ref{alg:trans-minmax} は各 $p\in P$ について,$t$ を含む節ごとに新規変数を導入し,最後に $(h\lor c_1)$ を追加する.
以下では,変数ごとに追加される新規変数と制約を明示する.\\
また,行26で各変数$p$に対して追加する節$(h \lor c_1)$を以後,極小(極大)性制約と呼ぶ.

%=========================
\subsubsection*{(1) 変数 $p$ に対する処理(min:$t=p,\ h=\neg p$)}

\noindent
$\Phi$ のうち $t=p$ を含む節は
\begin{center}
 $~~~~~~C_1=(p\lor q)$\\
 $\qquad C_2=(p\lor \neg q)$

\end{center}

の 2 つである.それぞれに対して新規変数を導入する

\begin{itemize}
  \item 節 $C_1=(p\lor q)$ に対して新規変数 $x_{p,1}$ を導入する.
  この節で $t=p$ 以外のリテラルは $q$ だけなので,
  $x_{p,1}$ は$q$ が偽であることを表す指示変数($x_{p,1}\leftrightarrow \neg q$)になる.
  Algorithm~\ref{alg:trans-minmax}(行18,22)に従い,次を $M$ に追加する
  \begin{align}
    (\neg x_{p,1}\lor \neg q)\ \land\ (q \lor x_{p,1}). \label{eq:add-xp1}
  \end{align}

  \item 節 $C_2=(p\lor \neg q)$ に対して新規変数 $x_{p,2}$ を導入する.
  $t=p$ 以外のリテラルは $\neg q$ だけなので,
  $x_{p,2}\leftrightarrow \neg(\neg q)\equiv q$ を表す($x_{p,2}\leftrightarrow q$).
  よって追加制約は
  \begin{align}
    (\neg x_{p,2}\lor \neg(\neg q))\ \land\ (\neg q \lor x_{p,2})
    \;\equiv\;
    (\neg x_{p,2}\lor q)\ \land\ (\neg q \lor x_{p,2}). \label{eq:add-xp2}
  \end{align}
\end{itemize}

\noindent
このとき,$c_1$ は,$p$ を含むどれかの節で,$p$ 以外がすべて偽を表す選言なので
\[
c_1 = x_{p,1}\ \lor\ x_{p,2}.
\]
最後に Algorithm~\ref{alg:trans-minmax}(行26)により極小性制約 $(h\lor c_1)$ を追加する
\begin{align}
(\neg p \lor x_{p,1}\lor x_{p,2}). \label{eq:add-p-main}
\end{align}

%=========================
\subsubsection*{(2) 変数 $q$ に対する処理(min:$t=q,\ h=\neg q$)}

\noindent
今度は $t=q$ を含む節だけを見る.$\Phi$ で $q$ を含むのは
\[
C_1=(p\lor q)
\]
のみである.したがって新規変数は 1 つ

\begin{itemize}
  \item 節 $C_1=(p\lor q)$ に対して新規変数 $x_{q,1}$ を導入する.
  $t=q$ 以外のリテラルは $p$ だけなので,$x_{q,1}\leftrightarrow \neg p$ を表す.
  追加制約は
  \begin{align}
    (\neg x_{q,1}\lor \neg p)\ \land\ (p \lor x_{q,1}). \label{eq:add-xq1}
  \end{align}
\end{itemize}

\noindent
このとき $c_1=x_{q,1}$ であり,極小性制約は
\begin{align}
(\neg q \lor x_{q,1}). \label{eq:add-q-main}
\end{align}

%=========================
\subsubsection*{(3) 変数 $r$ に対する処理(min:$t=r,\ h=\neg r$)}

\noindent
$\Phi$ で $r$ を含むのは
\[
C_3=(\neg p\lor r)
\]
のみである.新規変数を 1 つ導入する:

\begin{itemize}
  \item 節 $C_3=(\neg p\lor r)$ に対して新規変数 $x_{r,1}$ を導入する.
  $t=r$ 以外のリテラルは $\neg p$ なので,$x_{r,1}\leftrightarrow \neg(\neg p)\equiv p$ を表す.
  追加制約は:
  \begin{align}
    (\neg x_{r,1}\lor \neg(\neg p))\ \land\ (\neg p \lor x_{r,1})
    \;\equiv\;
    (\neg x_{r,1}\lor p)\ \land\ (\neg p \lor x_{r,1}). \label{eq:add-xr1}
  \end{align}
\end{itemize}

\noindent
このとき $c_1=x_{r,1}$ であり,極小性制約は:
\begin{align}
(\neg r \lor x_{r,1}). \label{eq:add-r-main}
\end{align}

%------------------------------------------------------------
\paragraph{得られる追加制約 $M$ と出力 $\Omega$}
以上で生成された追加制約 $M$ は
\begin{align}
M \;=\;&
(\neg x_{p,1}\lor \neg q)\land(q\lor x_{p,1})
\land(\neg x_{p,2}\lor q)\land(\neg q\lor x_{p,2})
\land(\neg p\lor x_{p,1}\lor x_{p,2})
\nonumber\\
&\land(\neg x_{q,1}\lor \neg p)\land(p\lor x_{q,1})
\land(\neg q\lor x_{q,1})
\nonumber\\
&\land(\neg x_{r,1}\lor p)\land(\neg p\lor x_{r,1})
\land(\neg r\lor x_{r,1}).
\label{eq:M-example}
\end{align}
したがって出力は $\Omega=\Phi\land M$ である:
\begin{align}
\Omega \;=\;& (p \lor q)\land(p \lor \neg q)\land(\neg p \lor r)\ \land\ M.
\label{eq:Omega-example}
\end{align}

導入された各新規変数は,対応する節において,注目リテラル $t$ 以外がすべて偽を表す指示変数である(Algorithm~\ref{alg:trans-minmax} 行18--22).
例えば
\[
x_{p,1}\leftrightarrow \neg q,\qquad
x_{p,2}\leftrightarrow q,\qquad
x_{q,1}\leftrightarrow \neg p,\qquad
x_{r,1}\leftrightarrow p
\]
となっており,極小性制約
\[
(\neg p \lor x_{p,1}\lor x_{p,2}),\quad
(\neg q \lor x_{q,1}),\quad
(\neg r \lor x_{r,1})
\]
は$p,q,r$ を真にする場合は,それが必須であることを $x$ によって証明しなければならないという意味を持ち,CNF に埋め込むことで極小性を与える.

%------------------------------------------------------------
\paragraph{極大変換(mode=max )}
極大変換では,各変数ごとに
\[
t\gets \neg p,\ \ h\gets p
\]
のように $t$ と $h$ が入れ替わるだけで,節の走査・新規変数の導入・制約生成は同一である
(Algorithm~\ref{alg:trans-minmax} 行5--11).
したがって,上の例をそのまま用い,
$\neg p$ を含む節だけが $p$ の処理対象になる点に注意して同様に展開できる.
%=========================


% \section{変換の実装}
% \label{sec:encoding}
% \cite{koshimura2009minimal}で提案された極大・極小モデルの計算方法,\cite{adachi2023}で提案された極小モデルの計算方法,さらに本論文での極大モデルの計算方法を実装するにあたり,どのように実装したかを図\ref{fig:archi}に示す.\\
% 今回の実装では,グラフ問題を用いて実装を行い,極大モデルは極大独立集合問題(MIS),極小モデルは極小支配集合問題(MDS)で計算した.また,CSPの構築にはcp4rust用い,SAT変換は順序符号化,SATソルバーはCaDiCaLを使用した.

% \begin{figure}[ht]
%   \centering
%   \resizebox{\textwidth}{!}{%
%     \begin{tikzpicture}[
  font=\normalsize,
  >=Stealth,
  node distance=14mm and 22mm,
  module/.style={
    rectangle, rounded corners,
    draw=blue!60, thick,
    fill=blue!8,
    align=center,
    minimum width=40mm,
    minimum height=14mm,
    inner sep=6pt
  },
  data/.style={
    rectangle, rounded corners,
    draw=green!60!black, thick,
    fill=green!10,
    align=center,
    minimum width=36mm,
    minimum height=14mm,
    inner sep=6pt
  },
  external/.style={
    rectangle, rounded corners,
    draw=orange!70!black, thick,
    fill=orange!12,
    align=center,
    minimum width=42mm,
    minimum height=14mm,
    inner sep=6pt
  },
  flow/.style={->, thick},
  altflow/.style={->, thick, dashed},
  loop/.style={->, thick, dotted}
]

% ============================================================
% 入力層
% ============================================================
\node[data] (dimacs) {DIMACS\\グラフ};
\node[module, right=of dimacs] (graph) {Graph\\隣接リスト};

\draw[flow] (dimacs) -- (graph);

% ============================================================
% 問題定式化層(左)
% ============================================================
\node[module, right=of graph, yshift=26mm] (mds) {MDS 定式化};
\node[module, right=of graph, yshift=-26mm] (mis) {MIS 定式化};

\draw[flow] (graph) -- (mds);
\draw[flow] (graph) -- (mis);

% ============================================================
% MDS 系(右上ブロック)
% ============================================================
\node[module, right=26mm of mds, yshift=-12mm] (mds_basic)
{MDS\\直接 CNF};

\node[module, right=26mm of mds, yshift=12mm] (mds_trans)
{MDS\\CSP 変換};

\draw[altflow]
  (mds) -- node[midway, above]
  {\cite{koshimura2009minimal}の変換}
  (mds_basic);
\draw[flow]    
  (mds) -- node[midway, above]
  {提案変換}
  (mds_trans);

% ============================================================
% MIS 系(右下ブロック)
% ============================================================
\node[module, right=26mm of mis, yshift=-12mm] (mis_basic)
{MIS\\直接 CNF};

\node[module, right=26mm of mis, yshift=12mm] (mis_trans)
{MIS\\CSP 変換};

\draw[altflow] 
  (mis) -- node[midway, above]
  {\cite{koshimura2009minimal}の変換}
  (mis_basic);
\draw[flow]    
  (mis) -- node[midway, above]
  {\cite{adachi2023}の変換}
  (mis_trans);

% ============================================================
% CSP 層
% ============================================================
\node[module, right=32mm of mds_trans, yshift=-26mm] (csp)
{CSP 構築\\(cp4rust)};

\draw[flow] (mds_trans) -- (csp);
\draw[flow] (mis_trans) -- (csp);

% ============================================================
% エンコード層
% ============================================================
\node[module, right=of csp] (encoder)
{SAT 変換\\OrderEncoderLe};

\draw[flow] (csp) -- (encoder);

% ============================================================
% SAT ソルバ層
% ============================================================
\node[external, below=24mm of encoder] (solver)
{CaDiCaL\\all\_solutions()};

\node[module, right=of solver] (blocking)
{ブロッキング節\\$\neg$ 現在解};

\draw[flow] (encoder) -- (solver);
\draw[flow] (mds_basic) -- (solver);
\draw[flow] (mis_basic) -- (solver);
\draw[loop] (solver) -- (blocking);
\draw[loop] (blocking) -- (solver);

% ============================================================
% 出力層
% ============================================================
\node[data, below=18mm of solver, xshift=-20mm] (solutions)
{解集合\\$\{v_1,v_2,\dots\}$};

\node[data, below=18mm of solver, xshift=+20mm] (stats)
{統計情報\\変数数 / 節数};

\draw[flow] (solver) -- (solutions);
\draw[flow] (solver) -- (stats);

% ============================================================
% レイヤ枠
% ============================================================
\begin{pgfonlayer}{background}
\node[draw=black!30, rounded corners, thick,
      fit=(dimacs)(graph),
      label=above:{入力層}] {};

\node[draw=black!30, rounded corners, thick,
      fit=(mds)(mis)(mds_basic)(mds_trans)(mis_basic)(mis_trans),
      label=above:{問題定式化層}] {};

\node[draw=black!30, rounded corners, thick,
      fit=(csp),
      label=above:{制約層(CSP)}] {};

\node[draw=black!30, rounded corners, thick,
      fit=(encoder),
      label=above:{エンコード層}] {};

\node[draw=black!30, rounded corners, thick,
      fit=(solver)(blocking),
      label=above:{SAT ソルバ層}] {};

\node[draw=black!30, rounded corners, thick,
      fit=(solutions)(stats),
      label=above:{出力層}] {};
\end{pgfonlayer}

\end{tikzpicture}
%
%   }
%   \caption{CSP--SAT に基づく極小・極大集合列挙アーキテクチャ}
%  \label{fig:archi}
% \end{figure}

%%%%%%%%%%%%%%%%%%%%%%%%%%%%%%%%%%%%%%%%%%%%%%%%%%%%%%%%%% 
\chapter{実行実験} \label{chap:experiment}
%%%%%%%%%%%%%%%%%%%%%%%%%%%%%%%%%%%%%%%%%%%%%%%%%%%%%%%%%% 
\section{実験概要}
論文\cite{adachi2023}では,SATソルバーを1回起動するだけで極小モデルを計算する手法が提案されたが,当該手法の実行性能に関する評価は十分に行われていなかった.同様に,本論文で提案する,SATソルバー1回起動による極大モデル計算方法についても,その性能は未評価である.そこで本節では,極小モデル計算の評価対象としてグラフ問題である極小支配集合問題(Minimal Dominating Set: MDS)を,極大モデル計算の評価対象としてグラフ問題である極大独立集合問題(Maximal Independent Set: MIS)を用いる.
具体的には,論文\cite{koshimura2009minimal}で提案された極大モデル計算手法(MIS-basic)および極小モデル計算手法(MDS-basic),論文\cite{adachi2023}で提案された極小モデル計算手法(MDS-trans),ならびに本論文で提案する極大モデル計算手法(MIS-trans)を比較対象とし,これらの手法に対して実行実験を行うことで性能評価を行う.\\

\section{実験環境}
本実験における実行環境は以下の通りである.
\begin{itemize}
 \item 実行環境:Mac mini,Apple M4,32GB
 \item SATソルバー:$CaDiCaL~3.0.0$ 
 \item 使用言語:$Rust$ 
 \item 制限時間:1問当たり30分
\end{itemize}

\section{ベンチマーク}
今回の実行実験で使用するベンチマークは3行n列($n=1,2,...,19$)のGrid graphの全19問である.
ここで,Grid graphとは,整数格子上の点を頂点とし,上下左右に隣接する点同士を辺で結んだグラフである.\\
支配集合とは,グラフ \(G=(V,E)\) において,頂点集合 \(D\subseteq V\) が,全ての頂点を自分自身または隣接頂点で覆うとき,\(D\) を支配集合と呼ぶ.すなわち,任意の頂点 \(v\in V\) について
\[
v\in D \ \ \text{または}\ \ \exists u\in D\ \text{s.t.}\ \{u,v\}\in E
\]
が成り立つ.\\
独立集合とは,グラフ \(G=(V,E)\) において,頂点集合 \(I\subseteq V\) が集合内の頂点同士が隣り合わない性質を満たすとき,\(I\) を独立集合と呼ぶ.これは,
\[
\forall u,v\in I\ (u\neq v)\Rightarrow \{u,v\}\notin E
\]
で表される.

\begin{example}\rm
 図\ref{fig:grid-mds}に3$\times$4のGrid graphにおける極小支配集合(MDS)の例を,図\ref{fig:grid-mis}に3$\times$4のGrid graphにおける極大独立集合(MIS)の例を示す.

 \begin{figure}[ht]
  \centering
    % tikz/gridgraph.tex
\begin{tikzpicture}[
  scale=1.0,
  v/.style={
    circle, draw,
    minimum size=7.2mm,
    inner sep=0pt,
    font=\normalsize,
    text height=1.6ex,
    text depth=.25ex
  },
  dom/.style={v, fill=red},                         % MDS vertex
  indep/.style={v, fill=green},      % MIS vertex
  both/.style={v, fill=gray!30, double, double distance=0.9pt}, % both
  edge/.style={draw, line width=0.65pt},
  domEdge/.style={draw, densely dashed, line width=0.9pt, dash pattern=on 2.2pt off 1.6pt}
]

% --- nodes (3x4) ---
\node[indep] (a) at (0,2) {a};
\node[v]   (b) at (1,2) {b};
\node[indep] (c) at (2,2) {c};
\node[v]     (d) at (3,2) {d};

\node[v]     (e) at (0,1) {e};
\node[indep] (f) at (1,1) {f};
\node[v]     (g) at (2,1) {g};
\node[indep]  (h) at (3,1) {h};

\node[indep]  (i) at (0,0) {i};
\node[v]     (j) at (1,0) {j};
\node[indep] (k) at (2,0) {k};
\node[v]     (l) at (3,0) {l};

% --- grid edges (solid) ---
\draw[edge] (a)--(b) (b)--(c) (c)--(d);
\draw[edge] (e)--(f) (f)--(g) (g)--(h);
\draw[edge] (i)--(j) (j)--(k) (k)--(l);

\draw[edge] (a)--(e) (e)--(i);
\draw[edge] (b)--(f) (f)--(j);
\draw[edge] (c)--(g) (g)--(k);
\draw[edge] (d)--(h) (h)--(l);

\end{tikzpicture}

  \caption{3$\times$4Grid graphのMIS}
  \label{fig:grid-mis}
 \end{figure}
\end{example}


 \begin{figure}[ht]
  \centering
    \input{tikz/gridgraph-mds}
  \caption{3$\times$4Grid graphのMDS}
  \label{fig:grid-mds}
 \end{figure}

\FloatBarrier

\section{CPU時間の比較}
表\ref{tab:mds-results}に MDSの既存手法と提案手法,表\ref{tab:mis-results}に MISの既存手法と提案手法の実行結果(各ベンチマークに対するCPU時間と得られた結果数)を示す.左に既存手法のCPU時間(秒)と得られた結果数,右に提案手法のCPU時間(秒)と得られた結果数を示す.各行において同一ベンチマークに対する両手法の計算時間結果を比較できる.また,赤文字で示されている結果は既存手法より短い時間で解けた問題である.MDS および MIS について,それぞれのカクタスプロットを図\ref{img:mds-plot},図\ref{img:mis-plot}に示す.横軸は Grid graph の列数 $n$,縦軸は各問題を解くのに要した CPU 時間である.同一の縦軸値に対して右側に位置するほど多くの問題を解けたことを意味し,下側に位置するほど短時間で解けたことを意味する.

図\ref{img:model-discovery}に SAT ソルバー起動に要したCPU時間とモデル計算に要したCPU時間の比較を示す.縦軸はCPU時間(秒),横軸は計算したモデルの数を示している.また,緑のプロットは既存手法でモデル計算に要した合計時間,赤のプロットは提案手法でモデル計算に要した合計時間,水色の棒グラフは既存手法においてSATソルバー起動でモデルを発見するために要した時間,青の棒グラフは既存手法においてSATソルバー起動でUNSATを出力するために要した時間である.

図\ref{img:model-times}に既存手法と提案手法のモデル計算に要したCPU時間推移の比較を示す.青のグラフが既存手法において各モデル計算に要した時間を,赤のグラフが提案手法において各モデル計算に要した時間を示している.このグラフでは上に行くほどモデル計算に時間を要したと言える.

まず表\ref{tab:mds-results}より,極小支配集合問題の全解列挙において既存手法は $3\times10$ まで解けたのに対し,提案手法は $3\times12$ まで解けた.さらに同一サイズの問題で比較すると,$3\times10$ において既存手法が 1361.24 秒を要したのに対し,提案手法は 3.620 秒で解けており,20 分以上の大幅な短縮が確認できる.

次に表\ref{tab:mis-results}より,極大独立集合問題の全解列挙において既存手法は $3\times13$ まで解けたのに対し,提案手法は $3\times18$ まで解けた.また $3\times13$ におけるCPU時間は,既存手法が 584.70 秒であるのに対し,提案手法は 0.36 秒であり,約 10 分の短縮が確認できる.

図\ref{img:mds-plot}および図\ref{img:mis-plot}においても,いずれの問題設定でも提案手法が既存手法よりも下側かつ右側に位置している.したがって,提案手法は既存手法に比べて,同一問題をより短時間で解けるだけでなく,制限時間内に解ける問題サイズの上限も拡大しており,総合的に高い性能を示すことが分かる.

また,図\ref{img:model-discovery}より,MDS および MIS のいずれにおいても,既存手法では頻繁な SAT ソルバーの起動がオーバーヘッドとなり,CPU時間全体の増加を招いていることが分かる.一方,提案手法ではソルバーの起動回数が大幅に抑制されているため,起動に伴うコストが低減され,効率的な計算が実現できている.

さらに,図\ref{img:model-times}より,提案手法は既存手法と比較して,各モデルを計算するのに要する時間がモデル数の増加に伴い短くなり,かつ安定して推移していることが読み取れる.これは,提案手法が大規模な解空間を持つ問題に対しても,高速に解を列挙し続けられることを示しており,CPU時間の短縮を裏付けるものである.

\begin{figure}[t]
  \centering

  %--- table ---
  \begin{minipage}{0.95\linewidth}
    \centering
    \captionof{table}{MDS のCPU時間と結果数}
    \label{tab:mds-results}
    \begin{tabular}{c|r|r|r|r}

\hline
& \multicolumn{4}{c}{極小モデル(極小支配集合)} \\
\hline
Benchmark & \multicolumn{2}{c|}{既存} & \multicolumn{2}{c}{提案} \\
\hline
& \multicolumn{1}{c|}{time} & \multicolumn{1}{c|}{results} & \multicolumn{1}{c|}{time} & \multicolumn{1}{c}{results}\\
\hline
\hline
grid3$\times$1  &    $<0.0$ &    2     &    \textcolor{red}{$<0.0$} &    2 \\
grid3$\times$2  &    $<0.0$ &    7     &    \textcolor{red}{$<0.0$} &    7 \\
grid3$\times$3  &    $<0.0$ &    16    &    \textcolor{red}{$<0.0$} &    16 \\
grid3$\times$4  &    $<0.0$ &    53    &    \textcolor{red}{$<0.0$} &    53 \\
grid3$\times$5  &    0.01 &    154   &    \textcolor{red}{$<0.0$} &    154 \\
grid3$\times$6  &    0.15 &    436   &    \textcolor{red}{0.01} &    436 \\
grid3$\times$7  &    1.86 &    1268  &    \textcolor{red}{0.03} &    1268 \\
grid3$\times$8  &    26.32 &    3660 &    \textcolor{red}{0.13} &    3660 \\
grid3$\times$9  &    182.35 &    10610 &    \textcolor{red}{0.58} &    10610 \\
grid3$\times$10 &    1361.24 &    30744 &    \textcolor{red}{3.620} &    30744 \\
grid3$\times$11 &     T.O. &    -   &    \textcolor{red}{24.84} &    89079 \\
grid3$\times$12 &     T.O. &    -   &    \textcolor{red}{207.90} &    258251 \\
grid3$\times$13 &     T.O. &    -   &     T.O. &    - \\
\hline
\end{tabular}


  \end{minipage}

  \vspace{6pt}

  %--- figure ---
  \begin{minipage}{0.95\linewidth}
    \centering
    \includegraphics[width=0.75\linewidth]{images/mds-plot.png}
    \captionof{figure}{MDS のCPU時間と結果数のプロット}
    \label{img:mds-plot}
  \end{minipage}
\end{figure}

\begin{figure}[t]
  \centering

  %--- table ---
  \begin{minipage}{0.95\linewidth}
    \centering
    \captionof{table}{MIS のCPU時間と結果数}
    \label{tab:mis-results}
    \begin{tabular}{c|r|r|r|r}

\hline
& \multicolumn{4}{c}{極大モデル(極大独立集合)} \\
\hline
Benchmark & \multicolumn{2}{c|}{既存} & \multicolumn{2}{c}{提案} \\
\hline
& \multicolumn{1}{c|}{time} & \multicolumn{1}{c|}{results} & \multicolumn{1}{c|}{time} & \multicolumn{1}{c}{results}\\
\hline
\hline
grid3$\times$1  &  $<0.0$ &  2     &  \textcolor{red}{$<0.0$} &  2 \\
grid3$\times$2  &  $<0.0$ &  4     &  \textcolor{red}{$<0.0$} &  4 \\
grid3$\times$3  &  $<0.0$ &  10    &  \textcolor{red}{$<0.0$} &  10 \\
grid3$\times$4  &  $<0.0$ &  18    &  \textcolor{red}{$<0.0$} &  18 \\
grid3$\times$5  &  $<0.0$ &  38    &  \textcolor{red}{$<0.0$} &  38 \\
grid3$\times$6  &  $<0.0$ &  74    &  \textcolor{red}{$<0.0$} &  74 \\
grid3$\times$7  &  0.01 &  148   &  \textcolor{red}{$<0.0$} &  148 \\
grid3$\times$8  &  0.01 &  290   &  \textcolor{red}{$<0.0$} &  290 \\
grid3$\times$9  &  0.04 &  578   &  \textcolor{red}{0.01} &  578 \\
grid3$\times$10 &  0.14 &  1156  &  \textcolor{red}{0.01} &  1156 \\
grid3$\times$11 &  0.36 &  2706  &  \textcolor{red}{0.05} &  2706 \\
grid3$\times$12 &  114.72 &  5518 &  \textcolor{red}{0.14} &  5518 \\
grid3$\times$13 &  584.70 &  11228 &  \textcolor{red}{0.36} &  11228 \\
grid3$\times$14 &  T.O. &  -   &  \textcolor{red}{1.52} &  22884 \\
grid3$\times$15 &  T.O. &  -   &  \textcolor{red}{5.26}&  46634 \\
grid3$\times$16 &  T.O. &  -   &  \textcolor{red}{23.53} &  94978 \\
grid3$\times$17 &  T.O. &  -   &  \textcolor{red}{110.72} &  193518 \\
grid3$\times$18 &  T.O. &  -   &  \textcolor{red}{756.59} &  394286 \\
grid3$\times$19 &  T.O. &  -   &  T.O. &  - \\
\hline
\end{tabular}


  \end{minipage}

  \vspace{6pt}

  %--- figure ---
  \begin{minipage}{0.95\linewidth}
    \centering
    \includegraphics[width=0.75\linewidth]{images/mis-plot.png}
    \captionof{figure}{MIS のCPU時間と結果数のプロット}
    \label{img:mis-plot}
  \end{minipage}
\end{figure}

\begin{figure}[t]
 \centering
 \begin{minipage}{0.95\linewidth}
  \centering
  \includegraphics[width=1.0\linewidth]{images/model_discovery.png}
  \caption{SATソルバー起動により要したCPU時間の比較}
  \label{img:model-discovery}
 \end{minipage}
 
 \vspace{10pt}

 \begin{minipage}{0.95\linewidth}
  \centering
  \includegraphics[width=1.0\linewidth]{images/model_times.png}
  \caption{既存手法と提案手法の計算したモデル数とCPU時間推移}
  \label{img:model-times}
  \end{minipage}
\end{figure}

\FloatBarrier

\section{CNFの変数,節の数とSATソルバー起動回数の比較}
表\ref{tab:mds-mis-cnf} に,Grid graph を CNF として定式化した際の変数数と節数を示す.varsが変数の数,clausesが節の数を示している.また,提案手法ではCNFを変換した後の変数,節の数が示されている.各行において同一ベンチマークに対する両手法の変数,節の数を比較できる.

表\ref{tab:c-solver}に,各計算手法における SAT ソルバー(CaDiCaL)の起動回数と,既存手法に対する提案手法の割合(\%)を示す.各行において同一ベンチマークに対する両手法のSATソルバー起動回数を比較できる.

表\ref{tab:mds-mis-cnf}より,MDSでの提案手法では,CNF の変換過程で Tseitin 変換を用いているため,補助変数が導入され,既存手法に比べて変数数が増加している.MIS の方が既存手法と提案手法の性能差が大きく現れている要因として,MDSでの提案手法では Tseitin 変換により CNF の構造が複雑化することでCPU時間が増加しやすい点が挙げられる.その結果として,MDS では変換によるCPU時間の高速化が MIS ほど顕著に表れにくいと考えられる.

表\ref{tab:c-solver}より,提案手法では,SAT ソルバーの起動回数が期待通り,得られた解の数と一致しており,解の列挙に対して無駄な起動が発生していないことが確認できる.具体例として $3\times8$ の Grid graph に着目すると,MDSの既存手法は計算過程で CaDiCaL を 17406 回起動しているのに対し,提案手法は 3660 回であり,提案手法は既存手法に比べて約21\%の起動回数で済んでいる.このように SAT ソルバー起動回数が減少していることは,ソルバー起動に伴うオーバーヘッドの削減につながるため,CPU時間の短縮に寄与していると考えられる.

\begin{table}[t]
 \centering
 \caption{MDS と MIS のCNFの変数数と節数の比較}
 \label{tab:mds-mis-cnf}
 \begin{tabular}{c|r|r|r|r|r|r|r|r}
\hline
& \multicolumn{4}{c|}{MDS(極小モデル)} & \multicolumn{4}{c}{MIS(極大モデル)} \\
\hline
& \multicolumn{2}{c|}{既存} & \multicolumn{2}{c|}{提案} & \multicolumn{2}{c|}{既存} & \multicolumn{2}{c}{提案} \\
\hline
Benchmark & vars & clauses & vars & clauses & vars & clauses & vars & clauses \\
\hline\hline
grid3$\times$1  & 3  & 3  & 10  & 16  & 3  & 2  & 3  & 5  \\
grid3$\times$2  & 6  & 6  & 26  & 60  & 6  & 7  & 6  & 13 \\
grid3$\times$3  & 9  & 9  & 42  & 110 & 9  & 12 & 9  & 21 \\
grid3$\times$4  & 12 & 12 & 58  & 160 & 12 & 17 & 12 & 29 \\
grid3$\times$5  & 15 & 15 & 74  & 210 & 15 & 22 & 15 & 37 \\
grid3$\times$6  & 18 & 18 & 90  & 260 & 18 & 27 & 18 & 45 \\
grid3$\times$7  & 21 & 21 & 106 & 310 & 21 & 32 & 21 & 53 \\
grid3$\times$8  & 24 & 24 & 122 & 360 & 24 & 37 & 24 & 61 \\
grid3$\times$9  & 27 & 27 & 138 & 410 & 27 & 42 & 27 & 69 \\
grid3$\times$10 & 30 & 30 & 154 & 460 & 30 & 47 & 30 & 77 \\
grid3$\times$11 & -  & -  & 170 & 510 & 33 & 52 & 33 & 85 \\
grid3$\times$12 & -  & -  & 186 & 560 & 36 & 57 & 36 & 93 \\
grid3$\times$13 & -  & -  & -   & -   & 39 & 62 & 39 & 101 \\
grid3$\times$14 & -  & -  & -   & -   & -  & -  & 42 & 109 \\
grid3$\times$15 & -  & -  & -   & -   & -  & -  & 45 & 117 \\
grid3$\times$16 & -  & -  & -   & -   & -  & -  & 48 & 125 \\
grid3$\times$17 & -  & -  & -   & -   & -  & -  & 51 & 133 \\
grid3$\times$18 & -  & -  & -   & -   & -  & -  & 54 & 141 \\
grid3$\times$19 & -  & -  & -   & -   & -  & -  & - & - \\
\hline
\end{tabular}

\end{table}

\begin{table}[t]
 \centering
 \caption{MDSとMISのSATソルバー起動回数の比較}
 \label{tab:c-solver}
 %========================
% MDS + MIS (merged)
%========================

\begin{tabular}{c|>{\raggedleft\arraybackslash}p{10mm} >{\raggedleft\arraybackslash}p{10mm}|r|>{\raggedleft\arraybackslash}p{10mm} >{\raggedleft\arraybackslash}p{10mm}|r}
\hline
& \multicolumn{3}{c|}{極小モデル(極小支配集合)} & \multicolumn{3}{c}{極大モデル(極大独立集合)} \\
\hline
\multicolumn{1}{c|}{Benchmark} &
\multicolumn{1}{c}{既存} &
\multicolumn{1}{c}{提案} &
\multicolumn{1}{c|}{割合[\%]} &
\multicolumn{1}{c}{既存} &
\multicolumn{1}{c}{提案} &
\multicolumn{1}{c}{割合[\%]} \\
\hline
 grid3$\times$1  & 7      & 2      & \textbf{28.6} & 6     & 2     & \textbf{33.3} \\
 grid3$\times$2  & 23     & 7      & \textbf{30.4} & 11    & 4     & \textbf{36.4} \\
 grid3$\times$3  & 62     & 16     & \textbf{25.8} & 23    & 10    & \textbf{43.5} \\
 grid3$\times$4  & 207    & 53     & \textbf{25.6} & 48    & 18    & \textbf{37.5} \\
 grid3$\times$5  & 594    & 154    & \textbf{25.9} & 101   & 38    & \textbf{37.6} \\
 grid3$\times$6  & 1743   & 436    & \textbf{25.0} & 203   & 78    & \textbf{38.5} \\
 grid3$\times$7  & 5502   & 1268   & \textbf{23.0} & 440   & 156   & \textbf{35.5} \\
 grid3$\times$8  & 17406  & 3660   & \textbf{21.0} & 907   & 321   & \textbf{35.3} \\
 grid3$\times$9  & 49864  & 10610  & \textbf{21.3} & 1848  & 651   & \textbf{35.2} \\
 grid3$\times$10 & 138754 & 30744  & \textbf{22.2} & 3808  & 1335  & \textbf{35.1} \\
 grid3$\times$11 & -      & 89079  & -    & 7896  & 2706  & \textbf{34.2} \\
 grid3$\times$12 & -      & 258251 & -    & 16127 & 5518  & \textbf{34.2} \\
 grid3$\times$13 & -      & -      & -    & 33281 & 11228 & \textbf{33.8} \\
 grid3$\times$14 & -      & -      & -    & -     & 22884 & -    \\
 grid3$\times$15 & -      & -      & -    & -     & 46634 & -    \\
 grid3$\times$16 & -      & -      & -    & -     & 94978 & -    \\
 grid3$\times$17 & -      & -      & -    & -     & 193518& -    \\
 grid3$\times$18 & -      & -      & -    & -     & 394286& -    \\
 grid3$\times$19 & -      & -      & -    & -     & -     & -    \\
\hline
\end{tabular}
\end{table}

% \begin{table}[htbp]
%  \centering
%  \caption{MDS のCPU時間と結果数}
%  \label{tab:mds-results}
%  \begin{tabular}{c|r|r|r|r}

\hline
& \multicolumn{4}{c}{極小モデル(極小支配集合)} \\
\hline
Benchmark & \multicolumn{2}{c|}{既存} & \multicolumn{2}{c}{提案} \\
\hline
& \multicolumn{1}{c|}{time} & \multicolumn{1}{c|}{results} & \multicolumn{1}{c|}{time} & \multicolumn{1}{c}{results}\\
\hline
\hline
grid3$\times$1  &    $<0.0$ &    2     &    \textcolor{red}{$<0.0$} &    2 \\
grid3$\times$2  &    $<0.0$ &    7     &    \textcolor{red}{$<0.0$} &    7 \\
grid3$\times$3  &    $<0.0$ &    16    &    \textcolor{red}{$<0.0$} &    16 \\
grid3$\times$4  &    $<0.0$ &    53    &    \textcolor{red}{$<0.0$} &    53 \\
grid3$\times$5  &    0.01 &    154   &    \textcolor{red}{$<0.0$} &    154 \\
grid3$\times$6  &    0.15 &    436   &    \textcolor{red}{0.01} &    436 \\
grid3$\times$7  &    1.86 &    1268  &    \textcolor{red}{0.03} &    1268 \\
grid3$\times$8  &    26.32 &    3660 &    \textcolor{red}{0.13} &    3660 \\
grid3$\times$9  &    182.35 &    10610 &    \textcolor{red}{0.58} &    10610 \\
grid3$\times$10 &    1361.24 &    30744 &    \textcolor{red}{3.620} &    30744 \\
grid3$\times$11 &     T.O. &    -   &    \textcolor{red}{24.84} &    89079 \\
grid3$\times$12 &     T.O. &    -   &    \textcolor{red}{207.90} &    258251 \\
grid3$\times$13 &     T.O. &    -   &     T.O. &    - \\
\hline
\end{tabular}


% \end{table}

% \begin{figure}[h]
%  \centering
%  \includegraphics[scale=0.6]{images/mds-plot.png}
%  \caption{MDS のプロット}
%  \label{img:mds-plot}
% \end{figure}



% \begin{table}[h]
%  \centering
%  \caption{MIS のCPU時間と結果数}
%  \label{tab:mis-results}
%  \begin{tabular}{c|r|r|r|r}

\hline
& \multicolumn{4}{c}{極大モデル(極大独立集合)} \\
\hline
Benchmark & \multicolumn{2}{c|}{既存} & \multicolumn{2}{c}{提案} \\
\hline
& \multicolumn{1}{c|}{time} & \multicolumn{1}{c|}{results} & \multicolumn{1}{c|}{time} & \multicolumn{1}{c}{results}\\
\hline
\hline
grid3$\times$1  &  $<0.0$ &  2     &  \textcolor{red}{$<0.0$} &  2 \\
grid3$\times$2  &  $<0.0$ &  4     &  \textcolor{red}{$<0.0$} &  4 \\
grid3$\times$3  &  $<0.0$ &  10    &  \textcolor{red}{$<0.0$} &  10 \\
grid3$\times$4  &  $<0.0$ &  18    &  \textcolor{red}{$<0.0$} &  18 \\
grid3$\times$5  &  $<0.0$ &  38    &  \textcolor{red}{$<0.0$} &  38 \\
grid3$\times$6  &  $<0.0$ &  74    &  \textcolor{red}{$<0.0$} &  74 \\
grid3$\times$7  &  0.01 &  148   &  \textcolor{red}{$<0.0$} &  148 \\
grid3$\times$8  &  0.01 &  290   &  \textcolor{red}{$<0.0$} &  290 \\
grid3$\times$9  &  0.04 &  578   &  \textcolor{red}{0.01} &  578 \\
grid3$\times$10 &  0.14 &  1156  &  \textcolor{red}{0.01} &  1156 \\
grid3$\times$11 &  0.36 &  2706  &  \textcolor{red}{0.05} &  2706 \\
grid3$\times$12 &  114.72 &  5518 &  \textcolor{red}{0.14} &  5518 \\
grid3$\times$13 &  584.70 &  11228 &  \textcolor{red}{0.36} &  11228 \\
grid3$\times$14 &  T.O. &  -   &  \textcolor{red}{1.52} &  22884 \\
grid3$\times$15 &  T.O. &  -   &  \textcolor{red}{5.26}&  46634 \\
grid3$\times$16 &  T.O. &  -   &  \textcolor{red}{23.53} &  94978 \\
grid3$\times$17 &  T.O. &  -   &  \textcolor{red}{110.72} &  193518 \\
grid3$\times$18 &  T.O. &  -   &  \textcolor{red}{756.59} &  394286 \\
grid3$\times$19 &  T.O. &  -   &  T.O. &  - \\
\hline
\end{tabular}


% \end{table}

% \begin{figure}[h]
%  \centering
%  \includegraphics[scale=0.6]{images/mis-plot.png}
%  \caption{MIS のプロット}
%  \label{img:mis-plot}
% \end{figure}



%%%%%%%%%%%%%%%%%%%%%%%%%%%%%%%%%%%%%%%%%%%%%%%%%%%%%%%%%% 
\section{おわりに} \label{chap:conclusion}
%%%%%%%%%%%%%%%%%%%%%%%%%%%%%%%%%%%%%%%%%%%%%%%%%%%%%%%%%%

本論文では,SAT 符号化に基づく CNF 変換を用い,SAT ソルバーを 1 回起動するだけで命題論理式の極小モデル・極大モデルを計算する枠組みを整理した.
極小モデルについては既存の変換手法を実装し,極大モデルについては負リテラルに着目した同様の発想による CNF 変換方法を新たに提案・実装した.

3 行 $n$ 列の Grid graph に基づくベンチマークを用いて,提案手法(trans)と従来の反復起動法(basic)の性能を比較した.
極小支配集合問題(MDS)では,basic が $3\times10$ までしか解けなかったのに対し,trans は $3\times12$ まで解くことができた.
極大独立集合問題(MIS)では,basic が $3\times13$ までであったのに対し,trans は $3\times18$ まで解くことができた.
いずれの問題設定においても,SAT ソルバーの起動回数が大幅に削減され,計算時間の短縮と解ける問題サイズの拡大が確認できた.
以上の結果から,SAT 符号化による CNF 変換が極小・極大モデル計算の実用性を高める有効なアプローチであることを示した.

今後の課題として,まず,グラフ問題以外の制約充足問題や計画問題など,異なる問題領域に対しても同様の傾向が成り立つかを検証する必要がある.
また,変換に伴う補助変数・節の増加による CNF サイズの増大が性能に与える影響を分析し,冗長制約の削減や Tseitin 変換の設計最適化などにより変換そのものを効率化する余地がある.
さらに,CaDiCaL 以外のソルバーでの再現性確認や,特定の変数集合のみを極小化・極大化する部分集合最適化への拡張など,極小・極大モデル計算をより実用的なものに発展させていくことが期待される.


% ここまで

%%%%%%%%%%%%%%%%%%%%%%%%%%%%%%%%%%%%%%%%%%%%%%%%%%%%%%%%%% 
\chapter*{謝辞}
%%%%%%%%%%%%%%%%%%%%%%%%%%%%%%%%%%%%%%%%%%%%%%%%%%%%%%%%%%

本研究の機会を賜り,丁寧かつ熱心にご指導頂きました名古屋大学大学院情報学研究科 番原 睦則教授,宋剛秀准教授に深く感謝申し上げます.また,日頃より様々なご助言を下さいました番原研究室の皆様に感謝申し上げます.


%%% Local Variables:
%%% mode: japanese-latex
%%% TeX-master: "paper"
%%% End:         % 謝辞

\bibliographystyle{jplain} % 参考文献スタイル
\bibliography{bachelor,aisat}    % 参考文献リスト

% 付録
%\input{appendix}

\end{document}
%%%%%%%%%%%%%%%%%%%%%%%%%%%%%%%%%%%%%%%%%%%%%%%%%%%%%%%%%%

%%% Local Variables:
%%% mode: japanese-latex
%%% TeX-master: t
%%% End:
